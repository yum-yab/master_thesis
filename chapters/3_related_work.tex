\chapter{Related Work}

\section{Climate simulation datasets}

General infos from \cite{mpige}:

\begin{itemize}
  \item 
	
\end{itemize}

\subsection{RCP Scenarios}

% \subsection{Questions arising about using climate simulation datasets}
% 
% \begin{itemize}
%   \item How many ensemble members are needed for a correct assessment?
%   \item How to sort them out? Random?
%   \item 
% \end{itemize}

\subsection{MPI-GE - The Max Planck Institute grand Ensemble}

General information about the future scenarios (all based on the \textit{rcp85} dataset available to me on the DKRZ cluster, I just assumed its the same for other scenarios. Maybe need to confirm this):

\begin{itemize}
  \item \textbf{Time}: The time axis is  compromised of 1128 values, which count the days since 01.01.2005. The first one is ~380, so it actually starts somewhere in 2006, and all of those values are roughly 30 days apart. This axis is part of every dataset, all stored as floats.
  \item \textbf{Lat}: Vector of 96 Float Elements ranging from roughly -88 to 88. Results in a resolution of 1.875° in North-South direction.
  \item \textbf{Lon}: Vector of 192 Float Elements ranging from roughly 0 to 358. Results in a resolution of 1.875° in East-West direction. 
  \item \textbf{Pressure Level} (plev): Is given for each dataset and consists of 26 Floats, ranging from 10 to 100,000. Unit is $Pa$.
  \item \textbf{Eastward/Northward Wind}: Given as Floats in the unit of $ms^{-1}$ per \textit{(lat, lon, time, plev)}. Each compromises the wind direction in one orthogonal direction. Eastward wind directory is named \textit{ua}, northward \textit{va}  
  \item \textbf{Specific Humidity}: Specific humidity is given as a float without value. Reason is the unit is actually kg moisture per kilogramm air, which cancels out in the end. Is given for each \textit{(lat, lon, time, plev)}. Directory name: \textit{hus}
  \item \textbf{Surface Wind Speed}: Given as float per \textit{(lat, lon, time, height)}, represents the wind speed in $ms^{-1}$ (no Vector!!) near the surface level. Directory Name: \textit{sfcWind}
  \item \textbf{Evaporation}: Given as a float and per \textit{(lat, lon, time)}, represents the evaporation flux. Unit is $\frac{kg}{m^2s}$, directory name \textit{evspsbl}
  \item \textbf{Preciptation}: Given either as normal or convective flux ($\frac{kg}{m^2s}$) per \textit{(lat, lon, time)}. Directory name \textit{pr, prc}.
  \item \textbf{Water Vapor Content}: Integrated over the colum, given per \textit{(lat, lon, time)}, just the water vapor content, no wind(vector) involved. Directory name: \textit{prw}.
	
\end{itemize}



In \cite{mpige} there is much information available:




\subsection{CMIP5 - Coupled Model Intercomparison Project}

In \cite{taylor2012overview_cmip5}


\section{Precipitation Literature}

\subsection{Saisonality in Precipitation variability}


The work of \citeauthor{precipitation_seasonality}

\section{Means of moisture transport}

\subsection{vertically integrated water vapor transport}

As proposed by \citeauthor{AProposedAlgorithmforMoistureFluxesfromAtmosphericRivers} in \cite{AProposedAlgorithmforMoistureFluxesfromAtmosphericRivers}, one way of measuring moisture ($p$) transport is by vertically integrating over the different pressure levels the zonal and meridional fluxes $\overline{pu}$ and $\overline{pv}$. 

An example of using this method can be found in \cite{Ayantobo2021IntegratedMT} with many more references why this method is working well for these kinds of approaches. 

Also this paper lists some other methods of moisture transportation which are also used:

\begin{enumerate}
  \item integrated water vapor distributions (see \cite{gimeno2014atmospheric_rivers_review})
  \item the lagrangian approach
  \item stable oxygen isotope investigation
\end{enumerate}

\subsubsection{Usages of IVT and differences}

In \cite{ralph2017dropsonde} they used a vector field of the IVT: $\int_{p_{low}}^{p_{max}} qV dp$, where $p$ is the pressure level, $q$ is the humidity and $V$ the horizontal vector.

In \cite{sousa2020north} they used a scalar field based on the euclidian norm of the vector field used by \cite{ralph2017dropsonde}.


In \cite{Ayantobo2021IntegratedMT} they also used the euclidian norm on a similar field like \cite{ralph2017dropsonde} to measure the impact of  ENSO on south-chinese weather.
% They used it to measure the moisture and displayed the first graphics of the work, illustrating how an IVT map looks at a drought and during a flood.
% Also maps of the mean IVT during each months, displaying the intense raining/wet summers and dry winters

\subsection{Moisture Budget}

\citeauthor{atmos13101694} showed in their report \cite{atmos13101694} the directions of moisture flux on the continent borders based on the big ERA5 reanalysis.
They measure the moisture based on a equation called the \textit{Moisture Budget}, which is based on multiple Faktors: 



It seems related to the IVT the other authors used, but utilizes the gradient and some other differences. The complete formula is:

$$
\frac{1}{g} \frac{\delta}{\delta t} \int^{P_s}_0 q dp = - \nabla \cdot \frac{1}{g} \int^{P_s}_0 (qv) dp + E - P
$$

With: 

\begin{enumerate}
  \item $p$ is the pressure, $P_s$ is the surface pressure
  \item $q$ is the specific humidity
  \item $v$ is the horizontal wind vector
  \item $E$ is the evaporation
  \item $P$ is the Precipitation
\end{enumerate}


In the actual analysis they used mostly other metrics:


\begin{enumerate}
  \item Vertically integrated Moisture Convergence (\textit{VIMC}): It is basically the gradient of the specific moisture in the air times the Wind vector
  \item $P$ is the precipitation 
  \item $E$ is the evaporation
\end{enumerate}

Furthermore they evaluateded the correlation between the moisture transport and the precipitation variability, which correlate to a significant extent.

\section{Pattern analysis}

\subsection{Empirical Orthogonal Functions}

See \cite{hannachi2007eof_review} for a big overview of EOF in atmospheric science.

See \cite{Ayantobo2021IntegratedMT} for an similar approach as we plan it, except it only focuses on the past.
They 
