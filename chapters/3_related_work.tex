\chapter{Related Work}

\section{Climate simulation datasets}

General infos from \cite{mpige}:

\begin{itemize}
  \item 
	
\end{itemize}

\subsection{RCP Scenarios}

\subsection{Questions arising about using climate simulation datasets}

\begin{itemize}
  \item How many ensemble members are needed for a correct assessment?
  \item How to sort them out? Random?
  \item 
\end{itemize}

\subsection{MPI-GE - The Max Planck Institute grand Ensemble}

In \cite{mpige} theere is much inforamtion available:

\subsection{CMIP5 - Coupled Model Intercomparison Project}

In \cite{taylor2012overview_cmip5}


\section{Precipitation Literature}

\subsection{Saisonality in Precipitation variability}


The work of \citeauthor{precipitation_seasonality}

\section{Means of moisture transport}

\subsection{vertically integrated water vapor transport}

As proposed by \citeauthor{AProposedAlgorithmforMoistureFluxesfromAtmosphericRivers} in \cite{AProposedAlgorithmforMoistureFluxesfromAtmosphericRivers}, one way of measuring moisture ($p$) transport is by vertically integrating over the different pressure levels the zonal and meridional fluxes $\overline{pu}$ and $\overline{pv}$. 

An example of using this method can be found in \cite{Ayantobo2021IntegratedMT} with many more references why this method is working well for these kinds of approaches. 
Also this paper lists some other methods of moisture transportation which are also used:

\begin{enumerate}
  \item integrated water vapor distributions (see \cite{gimeno2014atmospheric_rivers_review})
  \item the lagrangian approach
  \item stable oxygen isotope investigation
\end{enumerate}

Also the authors of \cite{Ayantobo2021IntegratedMT} make a twofold contribution to this work. 
Here it will be about the IVT of them. 
They used it to measure the moisture and displayed the first graphics of the woork, illustrating how an IVT map looks at a drought and during a flood.
Also maps of the mean IVT during each months, displaying the intense raining/wet summers and dry winters

\subsection{Moisture Budget}

\citeauthor{atmos13101694} showed in their report \cite{atmos13101694} the directions of moisture flux on the continent borders based on the big ERA5 reanalysis.
They measure the moisture based on a equation called the \textit{Moisture Budget}, which is based on multiple Faktors: 

\begin{enumerate}
  \item Vertically integrated Moisture Convergence ($VIMC$): It is basically the gradient of the specific moisture in the air times the Wind vector
  \item $P$ is the precipitation 
  \item $E$ is the evaporation
\end{enumerate}

Furthermore they evaluateded the correlation between the moisture transport and the precipitation variability, which correlate to a significant extent.

\section{Pattern analysis}

\subsection{Empirical Orthogonal Functions}

See \cite{hannachi2007eof_review} for a big overview of EOF in atmospheric science.

See \cite{Ayantobo2021IntegratedMT} for an similar approach as we plan it, except it only focuses on the past.
They 
