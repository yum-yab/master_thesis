\begin{titlepage}
  \thispagestyle{titlepage}
  \begin{center}
  {\LARGE \textbf{\myuniversity}}\\[5mm]
%   \begin{figure*}[h]
%   \hbox{}\hfill
%   \begin{minipage}[t]{9cm}
% 			 \centering
% 			 \includegraphics[width=4cm]{logos/uni-leipzig-logo.png}
%   \end{minipage}
%   \hfill\hbox{}
%   \end{figure*}
  {\Large \myschool}\\
  {\large \mydepartment}\\[20mm]
		%{\large\bf \titleOfThesis}\\[10mm]
		{\Large\bf \titleOfThesis}\\[10mm]
    {\LARGE \kindOfThesis}\\[10mm]
  %{\small Author:}\\[1mm]
  %{\large \myfirstname~\mylastname}\\[5mm]
  %{\small \today}\\[15mm]
  
  
% Needs to fit: http://studium.fmi.uni-leipzig.de/fileadmin/Studienbuero/documents/Formulare/Deckblatt_Dipl_BSc_MSc_Arbeit_.rtf
    
    
\vspace{5cm}
% \begin{tabular}{p{.6\textwidth}p{.5\textwidth}}
%     \multicolumn{1}{l}{submitted at \today}  & \multicolumn{1}{r}{submitted by} \\[3mm]
%     & \multicolumn{1}{r}{{\large \myfirstname~\mylastname}} \\
%     & \multicolumn{1}{r}{Informatik Bsc.}
% \end{tabular}




%   \vspace{1cm}
%   \renewcommand{\arraystretch}{.9}
%   \begin{tabular}{cc}
% 	\multicolumn{2}{c}{\small Advisers:} \\[1mm]
% 	{\small Supervisor} & {\small Supervisor} \\
% 	{\large Johannes Frey, M.Sc.} & {\large Dr.-Ing. Sebastian Hellmann} \\[2mm]
% 	{\small \myworkinggroup} 			 & {\small \myworkinggroup} \\
% 	{\small \mycompany} 			 & {\small \mycompany} 	\\
% 	{\small \mycompanystreet} 		 & {\small \mycompanystreet} \\
% 	{\small \myunizipcity}		 & {\small \myunizipcity} 
%   \end{tabular}
  
		\renewcommand{\arraystretch}{1}
  \end{center}
  
Leipzig, März 2021\hfill vorgelegt von \\[3mm]
\hspace*{\fill} {\large \myfirstname~\mylastname}\\
\hspace*{\fill} Studiengang Bachelor Informatik

\vspace{1cm}

{\large \textbf{Betreuende Hochschullehrer:}}

{\large Dr.-Ing. Sebastian Hellmann }\\
\mycompany/KILT

{\large Johannes Frey, M.Sc. }\\
\mycompany/KILT

{\large Dr. Eric Peukert}\\
Abteilung Datenbanken Universität Leipzig
  
\end{titlepage}

\thispagestyle{empty}
\vspace*{\fill}
\begin{minipage}{.95\textwidth}
\textbf{\mylastname,~\myfirstname:}\\
\emph{\titleOfThesis}\\
\kindOfThesis, \myuni \\
\myplace, \myyear.
\end{minipage}

\cleardoublepage

% =============================================================================

%%%%%%%%%%%%%%%%%%%%%%%%%%%%%%%%%%%%%%%%%%%%%%%%%%%%%%%%%%%%%%%%%%%%%%%%%%%%%
%%% Inhaltsverzeichnis
%%%%%%%%%%%%%%%%%%%%%%%%%%%%%%%%%%%%%%%%%%%%%%%%%%%%%%%%%%%%%%%%%%%%%%%%%%%%%

\setcounter{tocdepth}{2}
\tableofcontents

%\addcontentsline{toc}{chapter}{Abbildungsverzeichnis}
%\listoffigures

%\addcontentsline{toc}{chapter}{Tabellenverzeichnis}
%\listoftables

%\addcontentsline{toc}{chapter}{Algorithmenverzeichnis}
%\listofalgorithms
%\todo{Anleitung Algorithmen schreiben: \url{http://en.wikibooks.org/wiki/LaTeX/Algorithms}}

\cleardoublepage
% =============================================================================

\phantomsection
\section*{Abstract}
\addcontentsline{toc}{chapter}{Abstract}
Over the last years, a huge amount of work has been done to improve the ability of machines to utilize data on the Web. One approach is the Semantic Web, using ontologies as a way to make the knowledge of a domain machine-usable. Even though many ontologies were developed and published, a unified system to handle those has not surfaced, leaving consumers as well as publishers to deal with many uncertainties and challenges. 

This thesis presents DBpedia Archivo, an augmented ontology archive. It discovers, crawls, versions, and archives ontologies available on the Web. Each version of them is persisted on the DBpedia Databus. Additionally, Archivo augments the ontologies with different tests and features. 
The goals of Archivo are to provide a backup service for ontology-versions as well as to encourage publishers to follow best practices. 
For this Archivo rates the ontologies with a star system, making problems visible at a glance. A comparison to existing, similar systems is given.

\cleardoublepage
% =============================================================================


%\thispagestyle{plain}
%\phantom{.}
%\vspace{70mm}

%\begin{center}
%	\todo[inline]{This section is optional! It is basically an motivational cite for this work as it can be found in many books. Example is provided}
%	\textit{
%		\vspace{0.5cm}
%		The validation of clustering structures is \\
%		the most difficult and frustrating part of cluster analysis. \\ 
%		\vspace{0.5cm}
%		Without a strong effort in this direction, \\
%		cluster analysis will remain a black art 
%		accessible only to those \\
%		true believers who have experience and great courage.}
%\end{center}
%\begin{flushright}
%	\citet{Jain1988}
%	Anil K. Jain and Richard C. Dubes
%\end{flushright}

%\uselengthunit{mm}
%textwidth=\printlength{\textwidth}\\
%textheight=\printlength{\textheight}\\
%top=\printlength{\top}\\

%\cleardoublepage
% =============================================================================


%\thispagestyle{plain}
%\section*{Acknowledgements}
%\todo[inline]{This section is optional!}

%\cleardoublepage
% =============================================================================

