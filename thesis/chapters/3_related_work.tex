\chapter{Related Work}
\label{ch:related_work}

% This chapter builds on the foundation of Chapter \ref{ch:basics}, explaining what dataset is actually used, the reason for this and its properties. 
This section outlies the current state-of-the-art in the main parts of this thesis explained in Section \ref{sec:research_questions}: Quantifying Moisture (Transport), extracting spatio-temporal patterns, tracking their change over time and visualizing the uncertain results in the end.
% Furthermore, it summarizes the current state-of-science in quantifying and calculating (patterns of) moisture transport and the usage of it.  

\section{Motivation}

As explained in Chapter \ref{ch:intro}, the approach of this thesis is motivated by the approach of \citeauthor{vietinghoff_visual_2021} in \cite{vietinghoff_visual_2021} and the affiliated dissertation \cite{vietinghoffdiss}, which tackles the issue of detecting critical points in unstable scalar fields.
Hereby \cite{vietinghoff_visual_2021} analyzes the MPI GE \cite{maher_max_2019} from the 5th phase of CMIP, an ensemble simulation with 50 members. 
The goal was to find the probable centers of pressure high/lows in the NAO pattern (see Section \ref{sec:climate}) and to track their shift over time. 
They employed a sliding window approach, computing the dominant pattern (see Section \ref{sec:eof})  for each window and member, and determine the likely areas of critical points by merging the results of different memebers per timestep. 
The centers of mass of these critical areas are then tracked over time to visualize the shift of the pressure high and low. 
The results show that the patterns do change, and this change is more pronounced if climate change is stronger. 
Also, there is no significant change if the climate remains stable.


\section{Moisture Transport}

% This section should explain in what ways moisture transport can be quantified and used, give a few examples for each and maybe motivate  why we do it like we want to. 

To computationally extract any spatio-temporal patterns of moisture (transport), it first needs to be quantified in any way.
The variable from the MPI GE CMIP6 used for this task is the \textit{specific humidity}, which has no unit and is a float value between 0.0 and 1.0, denoting the percentage of water in the air at a specific gridpoint. 
The vast majority of literature regarding moisture transport use some form of vertically integrated humidity and the variants will be explained in the following section.  
% \todo{Create a table showing how different quantification were used in different algorithms}
A popular usage of these quantifications was to find a filamentary weather structure called \enquote{Atmospheric Rivers}\footnote{earlier or alternative name: \enquote{Tropospheric Rivers}}, a prominent way of water vapor transportation in the extratropic regions \cite{gimeno_atmospheric_2014}. 


The most straightforward way of quantifying moisture is \textbf{Vertically Integrated Water Vapor (IWV)} \cite{gimeno_atmospheric_2014, schluessel_atmospheric_1990, bao_interpretation_2006, neiman_meteorological_2008, zhao_lagrangian_2021}, which is essentailly the vertical integral of the specific humidity $q$ over the pressure levels $p$ from earth's surface $P_S$ to some upper limit in the atmosphere:


\begin{equation}
\label{eq:iwv}
IWV = \frac{1}{g} \int^{P_s}_0 q ~\diff p
\end{equation}

% \todo{Insert some examples of usages with references to the papers, indicating the usefulness}
% See Section \ref{sec:atmo-rivers} for further explanation. 



% \subsection{Vertically Integrated Water Vapor (IWV)}

% \subsection{Vertically Integrated Water Vapor Transport (IVT)}

Similar to Equation \ref{eq:iwv},  \citeauthor{zhu_proposed_1998} proposed in \cite{zhu_proposed_1998} to use \textbf{Vertical Integrated Moisture Transport (IVT)} for Atmospheric River detection. 
It is calculated by vertically integrating over the different pressure levels the zonal (along latitude lines) and meridional (along longitude lines) fluxes.
It became a popular metric for finding atmospheric rivers \cite{gimeno_atmospheric_2014}, sometimes alongside IWV \cite{eiras-barca_seasonal_2016}.
IVT has the unit $\frac{kg}{ms}$ and is usually defined with

\begin{equation}
\label{eq:ivt}  
\overrightarrow{IVT} = \frac{1}{g} \int^{P_s}_0 q ~\binom{u}{v} ~ \diff p
\end{equation}

or in a mathematically equivalent form \cite{fernandez_analysis_2003}.
Here $u$ and $v$ stand for the zonal and meridional components of the horizontal wind vector. 
While Equation \ref{eq:ivt} yields a vector field, the euclidian norm of said vector field 

\begin{equation}
\label{eq:ivtnorm}
\lVert IVT \rVert = \frac{1}{g} \sqrt{(\int^{P_s}_0 qu ~ \diff p)^2 + (\int^{P_s}_0 qv ~ \diff p)^2}  
\end{equation}

is also a popular choice in detecting atmospheric rivers \cite{sousa_north_2020, ramos_atmospheric_2016} and other use cases \cite{ayantobo_integrated_2022}.

% An example of using this method can be found in \cite{ayantobo_integrated_2022} with many more references why this method is working well for these kinds of approaches. 

% Also, this paper lists some other methods of moisture transportation which are also used

The IVT is also part of the atmospheric moisture budget \cite{yang_moisture_2022} (and similar in \cite{seager_mechanisms_2020}) given by 
\begin{equation}
\label{eq:moisture_budget}
\frac{1}{g} \frac{\delta}{\delta t} \int^{P_s}_0 q ~ \diff p = - \nabla \cdot \frac{1}{g} \int^{P_s}_0 q ~\binom{u}{v} \diff p + E - P
\end{equation}

With $E$ being the total evaporation and $P$ the precipitation. 
\citeauthor{yang_moisture_2022} showed in their report \cite{yang_moisture_2022} the directions of moisture flux and its evolution in the last three decades. The analysis was done for all continental borders based on the big ERA5 reanalysis.
The metrics used for this analysis were mostly the evaporation $E$, precipitation $P$ and the moisture transport convergence  $VIMC = \frac{1}{g} \int^{P_s}_0  \nabla \cdot q ~\binom{u}{v} \diff ~ p$ from Equation \ref{eq:moisture_budget}.

While the integration in the previous equations integrates from the surface to the outer border of the atmosphere (0 Pa), it is quite common to integrate up until the limit of 300 hPa \cite{ayantobo_integrated_2022, zhu_proposed_1998, kim_ensos_2015, guirguis_circulation_2018}, since the amount of moisture in the regions from 300 hPa to 0 Pa is quite neglible and amounts in total to about 2-3 cm/year in terms of freshwater flux \cite{zhou_atmospheric_2005}.

There are also some other notable other algorithms, namely stable oxygen isotope investigation \cite{ma_atmospheric_nodate} and langragian backwards trajectories \cite{zhao_lagrangian_2021}, but both rather look for the origin of the  water vapor instead of its destination and are therefor out of scope for this thesis.

% \subsubsection{Usages of IVT and differences}
%
% In \cite{ralph_dropsonde_2017} they used a vector field of the IVT: $\int_{p_{low}}^{p_{max}} qV dp$, where $p$ is the pressure level, $q$ is the humidity and $V$ the horizontal vector.
%
% In \cite{sousa_north_2020} they used a scalar field based on the euclidian norm of the vector field used by \cite{ralph_dropsonde_2017}.
%
%
% In \cite{ayantobo_integrated_2022} they also used the euclidian norm on a similar field like \cite{ralph_dropsonde_2017} to measure the impact of  ENSO on south-chinese weather.
% % They used it to measure the moisture and displayed the first graphics of the work, illustrating how an IVT map looks at a drought and during a flood.
% % Also maps of the mean IVT during each months, displaying the intense raining/wet summers and dry winters
%
% % \subsection{Moisture Budget}
%
%
% Furthermore, they evaluated the correlation between the moisture transport and the precipitation variability, which correlate to a significant extent.
%
% %\section{Atmospheric Rivers}
% %\label{sec:atmo-rivers}
%
% \todo{I don't know where to put this, maybe it should go into the preliminaries}
%
% This section should explain atmospheric rivers, but since we don't know if they are even relevant so i write it in the end. 

\section{Pattern analysis regarding IVT}
\label{sec:related_pattern_analysis}

% \usepackage{tabularray}




While there are many areas of interest for the application of EOF, this Section will give an overview what kind of pattern analysis has been performed in relation with IVT data. 
An overview of datasets, timescopes and other metadata is given in Table \ref{tab:ivtpatterns-overview}.


\begin{table}
\centering
\caption{Overview table of patterns with moisture transport}
\tabcolsep=0.11cm
\scalebox{0.5}{
\begin{tblr}{
  cell{2}{2} = {r},
  cell{3}{2} = {r},
  cell{4}{2} = {r},
  cell{5}{2} = {r},
  cell{6}{2} = {r},
  cell{7}{2} = {r},
  cell{8}{2} = {r},
  cell{9}{2} = {r},
  cell{10}{2} = {r},
  cell{11}{2} = {r},
  cell{12}{2} = {r},
  vline{9} = {1,4,7-8,10}{},
  hline{2} = {-}{},
}
\label{tab:ivtpatterns-overview}
\textbf{Paper ID}                & \textbf{Release Year} & \textbf{Pattern extraction} & \textbf{Area of Interest} & \textbf{Timescope} & \textbf{Time Resolution} & \textbf{Studied Season} & \textbf{Variable used for EOF} \\
\cite{teale_patterns_2020}       & 2020                  & SOMs                        & USA east                  & 1979 to 2017       & daily                    & all year                & IVT norm                       \\
\cite{ayantobo_integrated_2022}  & 2022                  & EOF                         & China                     & 1979 to 2010       & daily                    & all year                & IVT norm                       \\
\cite{salstein_modes_1983}       & 1982                  & EOF                         & Northern hemishpere       & 1958 to 1973       & monthly/yearly           & all year                & IWV, IVT\_u IVT\_v, combined   \\
\cite{fernandez_analysis_2003}   & 2003                  & EOF                         & mediterranian sea         & 1948 to 1996       & 6hr                      & DJF                     & P                              \\
\cite{zhou_atmospheric_2005}     & 2005                  & EOF                         & China                     & 1951 to 1999       & monthly                  & JJA                     & P                              \\
\cite{guirguis_circulation_2018} & 2018                  & EOF                         & USA (west coast)          & 1948 to 2017       & daily                    & NDJF                    & IVT norm (assumed)             \\
\cite{kim_ensos_2015}            & 2014                  & EOF                         & western USA               & 1979 to 2010       & 6hr                      & DJF                     & IVT norm (assumed)             \\
\cite{zou_interdecadal_2018}     & 2018                  & EOF                         & TEIOWP                    & 1961 to 2015       & monthly                  & JJA                     & IVT                            \\
\cite{zou_investigating_2020}    & 2020                  & EOF                         & TEIOWP                    & 1958 to 2018       & 6hr/monthly              & JJA                     & Integrated Water Vapor Sink    \\
\cite{yao_simulation_2013}       & 2013                  & EOF                         & East Asia                 & 1997 to 2002       &                          & JJA                     & IVT\_u IVT\_v                  \\
\cite{li_quasi-4-yr_2012}        & 2012                  & EOF                         & East Asia                 & 1979 to 2009       & monthly                  & summer                  & IVT                            
\end{tblr}
}
\end{table}

Although most found related work uses EOF analysis, \citeauthor{teale_patterns_2020} employ an approach using Self Organizing Maps (SOMs) to detect patterns of moisture tronsport in the eastern United States.
SOMs are a machine learning approach to reduce data dimensionality, producing a 2D map of higher dimensional data. \todo{CITE?}
While they acknowledge the efficiency of EOF to extract dominant patterns, they emphasize the problem of required orthogonality of modes, which is not given for SOMs. 
The results show that fluxes with the highest moisture content occur less frequently than those with less moisture. 
But despite the higher moisture content, fluxes with lower moisture transport dominate water vapor movement due to their prevalence. 
Many of these fluxes meet typical criteria for atmospheric rivers, with varying trajectories and sources suggesting diverse mechanisms of formation. 
The temporal variability in monthly flux frequencies correlates with regional precipitation patterns, indicating that this approach is a valuable framework for studying precipitation changes \cite{teale_patterns_2020}.



\citeauthor{ayantobo_integrated_2022} analysed the primary six modes of EOF in China, which was grouped in deifferent regions for comparison. 
While the variances of IVT in eastern to southern China were quite high, the variances in northern China were quite low. 
It was shown by comparing the temporal patterns of the primary mode of EOF with the ENSO, that these patterns were related. 
The cross-wavelet coherence revealed that IVT and ENSO time-series were coherent, which implies that increased IVT was prevalent linked to increased ENSO activities \cite{ayantobo_integrated_2022}. 


Published in 1982, \citeauthor{salstein_modes_1983} provided the first example of calculating EOF on IVT. 
Based on data from 91 weather stations, they computed the IVT of the whole northern hemisphere. 
Statistical siginificance was determined by employing a Monte Carlo testing method. 
EOF was computed on the IWV, the zonal and meridional IVT fields respectively, but they also evaluated an approach of combining both IVT components in one data vector. 
They reported the significance of the primary mode of IWV, encoding nearly half (44 \%) of variance of the data. 


\citeauthor{fernandez_analysis_2003} analyzed the precipitation modes in the mediterranean sea and linking them to the moisture transport in the same area. 
A goal of this analysis was to contribute to the understanding of the reduction of precipitation which happened in the area as well as to the low-frequency precipitation variability, leading to multiyear drought periods.
They employed multiple methods of validating their data: The precipitation data as well as the wind/moisture data for IVT were validated with data from actual weather stations. 
The stability of the eigenvectors was tested with a Monte Carlo simulation, comparing the variability of actual data with random test data, while degeneracy of the EOF modes was tested using the method of \citeauthor{north_sampling_1982} \cite{north_sampling_1982}. 
Results of the analysis identify the interpretation of the three main precipitation modes:
The first mode (22 \% variance) seems to be linked to the NAO and Atlantic Storm tracks and associated moisture transports, while the second mode (16 \%) represents the internal redistribution of moisture in the mediterranean basin between the estern and western parts.  
The third mode (11 \%) explains increased precipitation in the northern part of the domain. 
Additionally, moisture transport during positive and negative phases of leading mode showed increased inflow of moisture from the west \cite{fernandez_analysis_2003}. 


Similar to \cite{fernandez_analysis_2003}, \citeauthor{zhou_atmospheric_2005} analyzed the anoumalous summer rainfall patterns over China and link them to water vapor transport. 
They confirmed their results by using a second dataset for IVT calculation. 
They showed that the primary mode of anomalous rainfall is associated with heavier rainfall in the Yangtze river region, while the same applies to the second mode and the Huaihe river. 
Connecting these patterns to moisture transport, they identified the different ways how these heavier rain areas are coming about by cerain convergences of water vapor transports. 
Furthermore they compared the supply of anoumalous rainfall patterns to the one of normal monsoon rainfall, revealing that those differ significantly \cite{zhou_atmospheric_2005}.


In \cite{guirguis_circulation_2018}, the authors calculate rotated EOF on IVT data and try to analyze the relation between the 15 most dominant modes and the occurence of atmospheric rivers (AR) on the USA west coast. 
For this they divided the coast into different regions and linked the activity (positive and negative) of the corresponding temporal pattern of each mode to the occurence of atmospheric rivers. 
It was found that a few modes seem very influational for certain reagions' AR activity, while others seem to play no role at all. 
They also identfied favorable and unfavorable circulation states (e.g. amongst others a low pressure anomaly in a certain region) for AR occurce \cite{guirguis_circulation_2018}. 


\citeauthor{kim_ensos_2015} showed in their analysis the connection of the IVT patterns in the western USA to three different ENSO events (eastern pacific El Niño (EPEN), central pacific El Niño (CPEN) and La Niña (NINA)). 
While EPEN events are associated with large positive IVT anomalies from the subtropical Pacific to the north-western USA, CPEN events lead to enhanced moisture transport to the southern USA. 
During NINA events the mean IVT anomaly is flipped in comparison to EPEN and CPEN. 
Furthermore it was shown that IVT patterns computed for these events differ significantly from the ones computed for neutral years.
Furthermore the results were connected to precipitation anomalies on the USA west coast, showing huge differences (especially for the northern part of the coast) for EPEN and CPEN events. 
But the authors also emphasize that while the suggestions are strong, exceptions occur (e.g. one El Niño leading to a dry winter, another to the opposite) and need to be studied in greater detail.

Similar to \cite{vietinghoff_visual_2021} and the approach of this thesis, \citeauthor{zou_interdecadal_2018} applied a sliding window approach to IVT patterns in the tropical Indian Ocean–western Pacific to analyze the evolution over time. 
For the studied period from 1961 to 2015, they studied every 20 year period with a 5 year sliding window, computing Multivariate EOFs for each window, resulting in vector fields of patterns. 
The results show that the two most significant modes show significant changes in the mid 80s: The primary mode is characterized by a anti-cyclonic pattern in the north-western Pacific, which shifts significantly to the south. 
An analysis of the relation to sea surface temperature (SST) revealed that the correlation between the mode and SST rose in the mid 80s, from weakly correlated to significant positive correlation between IVT and SST anomalies. 
Furthermore, the primary mode seems to be regulated significantly bei ENSO. 
The second most significant mode is related to the variablility of the tropical Indian Ocean dipole (defined by the differences in average SST) \cite{zou_interdecadal_2018}.

A different approach was employed by \cite{zou_investigating_2020}, evaluating the EOF patterns of vertically integrated apparent moisture sink. 
Results indicate that the primary mode is a southwest-northeast oriented dipole, while the secondary mode is a southwest-northeast oriented tripole. 
The primary mode seems to be heavily regulated by the ENSO in the previous winter season, while the second mode seems to originates from internal atmopheric variability. 
Based on the much higher standard deviations in ENSO years, it seems that water vapor source and sink tend to be dominated by the primary mode in ENSO years, while the secondary mode is prevalent in non-ENSO years. 


While the main focus of \cite{yao_simulation_2013} is to evaluate and compare a regional air-sea coupled model, they also performed EOF analysis on the zonal and meridional components of IVT, respectively. 
They used the results to evaluate the connection to SST, revealing that the results from the regional coupled model aligns better with results from other datasets and reality than the regional uncoupled model. 


\citeauthor{li_quasi-4-yr_2012} evaluated the connection of the IVT-EOF patterns to ENSO in the asian western northern Pacific.
They used a different approach then most in applying EOF to IVT, by concatenating the meridional and zonal components in one matrix and calculating EOF on it. 
To confirm their results, they compared the results with another reanalysis from the same (and a larger) region. 
Furthermore, these IVT patterns were linked to the SST. 
They revealed the characteristics of the two most significant modes, but most prominently they showed the quasi-4-year coupling of the two most prominent modes with ENSO \cite{li_quasi-4-yr_2012}. 






\section{Uncertainity Visualisation}

Since the used dataset (see Chapter \ref{ch:dataset}) is an ensemble simulation consisting of 50 members, most of the figures and other visual representations in this thesis need to display the uncertainity stemming from them. 
This section summarizes advances fitting for this topic, giving a frame of references of current possibilities of vizualizing uncertainity.

\citeauthor{kamal_recent_2021} give a recent overview over the whole topic of uncertainity visualization: From the introduction to  the whole concept of uncertainity, to the differentiation between different kinds of uncertainity in the visualisation process.
They grouped all kinds of representing uncertainity in two categories: quatification, consisting of mostly mathematical approaches of handling uncertain data, and visualisation, displaying the uncertain data in a way directly. 
An overview of the different kinds of uncertainity visualization were given: Manipulation of attributes (like shading), animation, visual variables (like color, hue, brightness), graphical techniques like box/scatter plots and glyphs. 
Furthermore, recent advances in uncertainity visualization are given, with a special emphasis on ensemble (simulation) data, big data and machine learning, listing the most prominent areas where the presentation of uncertainity is crucial. 
In the end, a framework for evaluating uncertainity visualization is presented, followed by an overview of possible future research directions \cite{kamal_recent_2021}. 


A way of using animation to display uncertainity in scalar fields was shown by \citeauthor{coninx_visualization_2011}. 
Their goal was to enrich the usual display of scalar fields with colormaps with additional uncertainity information. 
The tool of choice here was animated Perlin noise, and the uncertainity was presented by modifying the noise mask with the uncertainity information at each point. 
The results were tested using a psychophysical evaluation of contrast sensitivity thresholds \cite{coninx_visualization_2011}, evaluating effective parameters for proper presentation of the uncertain area \cite{coninx_visualization_2011}.

\citeauthor{sanyal_noodles_2010} proposed Noodles, a tool for displaying uncertainity in weather ensemble simulations. 
It employs three different ways of displaying uncertain isocontours: ribbon, glyphs and spaghetti plots. 
Additionally, they added tools for exploring the uncertainity of datasets, like an colormap of the whole dataset uncertainity.  
Uncertainity in spaghetti plots is clear (one line per member), but gets confusing and chaotic quickly. 
The glyphs display the uncertainity by different sizes,  and can be displayed on the whole map or alogside means of isocontours. 
Ribbons condense the information of multiple lines by adapting the ribbon width to the uncertainity of isolines at a specific gridpoint. 
The resulting tool was tested by two meteorologists, and classified the results as beneficial \cite{sanyal_noodles_2010}.  


Another way of visualizing groups of isocontours are contour boxplots proposed in \cite{whitaker_contour_2013}, grouping isocontours together in a similar way like conventional boxplots. 
This means that the easiest default presentation (spaghetti plots) is replaced by popular boxplot stats: The median, the mean, the quartiles around that mean, the whole range and the outliers (not part of the whole range). 
But the implementation is not as straight forward as in conventional boxplots. 
To quantify the aforementioned statistics, \citeauthor{whitaker_contour_2013} propose a data depth based approach, which encodes how much a particular sample is centrally located in its function (or in this case: How central is a isocontour to a whole set of isocontours). 
While the results look very promising, it lacks a publicly available implementation, making it hard to use the approach.

%
% Explain some usages of EOF in data, but extremely important: Explain what \cite{ayantobo_integrated_2022} did since its quite similar. 
%
% % \subsection{Empirical Orthogonal Functions}
%
% See \cite{hannachi_empirical_2007} for a big overview of EOF in atmospheric science.
%
% See \cite{ayantobo_integrated_2022} for a similar approach as we plan it, except it only focuses on the past.
% They 
