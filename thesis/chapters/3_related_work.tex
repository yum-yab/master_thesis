\chapter{Related Work}
\label{ch:related_work}

\section{Climate simulation datasets}


This section should explain what datasets are available and why I chose the MPI-GE CMIP6 \cite{olonscheck_new_2023}

Maybe but the comparison table from \cite{olonscheck_new_2023} here and expand it a bit. 



% General infos from \cite{maher_max_2019}:
% 
% \begin{itemize}
%   \item 
% 	
% \end{itemize}
% 
% \subsection{RCP Scenarios}
% 
% % \subsection{Questions arising about using climate simulation datasets}
% % 
% % \begin{itemize}
% %   \item How many ensemble members are needed for a correct assessment?
% %   \item How to sort them out? Random?
% %   \item 
% % \end{itemize}
% 
% \subsection{MPI-GE - The Max Planck Institute grand Ensemble}
% 
% General information about the future scenarios (all based on the \textit{rcp85} dataset available to me on the DKRZ cluster, I just assumed its the same for other scenarios. Maybe need to confirm this):
% 
% \begin{itemize}
%   \item \textbf{Time}: The time axis is  compromised of 1128 values, which count the days since 01.01.2005. The first one is ~380, so it actually starts somewhere in 2006, and all of those values are roughly 30 days apart. This axis is part of every dataset, all stored as floats.
%   \item \textbf{Lat}: Vector of 96 Float Elements ranging from roughly -88 to 88. Results in a resolution of 1.875° in North-South direction.
%   \item \textbf{Lon}: Vector of 192 Float Elements ranging from roughly 0 to 358. Results in a resolution of 1.875° in East-West direction. 
%   \item \textbf{Pressure Level} (plev): Is given for each dataset and consists of 26 Floats, ranging from 10 to 100,000. Unit is $Pa$.
%   \item \textbf{Eastward/Northward Wind}: Given as Floats in the unit of $ms^{-1}$ per \textit{(lat, lon, time, plev)}. Each compromises the wind direction in one orthogonal direction. Eastward wind directory is named \textit{ua}, northward \textit{va}  
%   \item \textbf{Specific Humidity}: Specific humidity is given as a float without value. Reason is the unit is actually kg moisture per kilogramm air, which cancels out in the end. Is given for each \textit{(lat, lon, time, plev)}. Directory name: \textit{hus}
%   \item \textbf{Surface Wind Speed}: Given as float per \textit{(lat, lon, time, height)}, represents the wind speed in $ms^{-1}$ (no Vector!!) near the surface level. Directory Name: \textit{sfcWind}
%   \item \textbf{Evaporation}: Given as a float and per \textit{(lat, lon, time)}, represents the evaporation flux. Unit is $\frac{kg}{m^2s}$, directory name \textit{evspsbl}
%   \item \textbf{Preciptation}: Given either as normal or convective flux ($\frac{kg}{m^2s}$) per \textit{(lat, lon, time)}. Directory name \textit{pr, prc}.
%   \item \textbf{Water Vapor Content}: Integrated over the colum, given per \textit{(lat, lon, time)}, just the water vapor content, no wind(vector) involved. Directory name: \textit{prw}.
% 	
% \end{itemize}
% 
% 
% 
% In \cite{maher_max_2019} there is much information available:




\section{Moisture Transport}

This section should explain in what ways moisture transport can be quantified and used, give a few examples for each and maybe motivate  why we do it like we want to. 

To computationally study the change of moisture transport it first needs to be quantified. 
The vast majority of literature use some form of vertically integrated humidity, the variants will be explained in the following section.  \todo{Create a table showing how different quatifications were used in different algorithms}
The main usage of these algorithms was to find a filamentary structure called \enquote{Atmospheric Rivers}, a prominent way of water vapor transportation in the extratropic regions \cite{gimeno_atmospheric_2014}. 

See Section \ref{sec:atmo-rivers} for further explaination. 


There are also some notable other algorithms, namely stable oxygen isotope investigation \cite{ma_atmospheric_nodate} and langragian backwards trajectories \cite{zhao_lagrangian_2021}, but both rather look for the origin of the WV instead of its destination and are therefor out of scope for this thesis.

\subsection{Vertically Integrated Water Vapor (IWV)}

\subsection{Vertically Integrated Water Vapor Transport (IVT)}

As proposed by \citeauthor{zhu_proposed_1998} in \cite{zhu_proposed_1998}, one way of measuring moisture ($p$) transport is by vertically integrating over the different pressure levels the zonal and meridional fluxes $\overline{pu}$ and $\overline{pv}$. 

An example of using this method can be found in \cite{ayantobo_integrated_2022} with many more references why this method is working well for these kinds of approaches. 

Also this paper lists some other methods of moisture transportation which are also used

\begin{enumerate}
  \item integrated water vapor distributions
  \item the lagrangian approach
  \item stable oxygen isotope investigation
\end{enumerate}

\subsubsection{Usages of IVT and differences}

In \cite{ralph_dropsonde_2017} they used a vector field of the IVT: $\int_{p_{low}}^{p_{max}} qV dp$, where $p$ is the pressure level, $q$ is the humidity and $V$ the horizontal vector.

In \cite{sousa_north_2020} they used a scalar field based on the euclidian norm of the vector field used by \cite{ralph_dropsonde_2017}.


In \cite{ayantobo_integrated_2022} they also used the euclidian norm on a similar field like \cite{ralph_dropsonde_2017} to measure the impact of  ENSO on south-chinese weather.
% They used it to measure the moisture and displayed the first graphics of the work, illustrating how an IVT map looks at a drought and during a flood.
% Also maps of the mean IVT during each months, displaying the intense raining/wet summers and dry winters

\subsection{Moisture Budget}

\citeauthor{yang_moisture_2022} showed in their report \cite{yang_moisture_2022} the directions of moisture flux on the continent borders based on the big ERA5 reanalysis.
They measure the moisture based on a equation called the \textit{Moisture Budget}, which is based on multiple Faktors: 



It seems related to the IVT the other authors used, but utilizes the gradient and some other differences. The complete formula is:

$$
\frac{1}{g} \frac{\delta}{\delta t} \int^{P_s}_0 q dp = - \nabla \cdot \frac{1}{g} \int^{P_s}_0 (qv) dp + E - P
$$

With: 

\begin{enumerate}
  \item $p$ is the pressure, $P_s$ is the surface pressure
  \item $q$ is the specific humidity
  \item $v$ is the horizontal wind vector
  \item $E$ is the evaporation
  \item $P$ is the Precipitation
\end{enumerate}


In the actual analysis they used mostly other metrics:


\begin{enumerate}
  \item Vertically integrated Moisture Convergence (\textit{VIMC}): It is basically the gradient of the specific moisture in the air times the Wind vector
  \item $P$ is the precipitation 
  \item $E$ is the evaporation
\end{enumerate}

Furthermore they evaluateded the correlation between the moisture transport and the precipitation variability, which correlate to a significant extent.

\section{Atmospheric Rivers}
\label{sec:atmo-rivers}

\todo{I dont know where to put this, maybe it should go into the preliminaries}

This section should explain atmospheric rivers, but since we dont know if they are even relevant so i write it in the end. 

\section{Pattern analysis}

Explain some usages of EOF in data, but extremely important: Explain what \cite{ayantobo_integrated_2022} did since its quite similar. 

\subsection{Empirical Orthogonal Functions}

See \cite{hannachi_empirical_2007} for a big overview of EOF in atmospheric science.

See \cite{ayantobo_integrated_2022} for an similar approach as we plan it, except it only focuses on the past.
They 
