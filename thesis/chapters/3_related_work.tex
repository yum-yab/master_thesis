\chapter{Related Work}
\label{ch:related_work}

This chapter builds on the foundation of Chapter \ref{ch:basics}, explaining what dataset is actually used, the reason for this and its properties. 
Furthermore, it summarizes the current state-of-science in quantifying and calculating (patterns of) moisture transport and the usage of it.  


\section{Uncertainty Visualisation}

\section{Moisture Transport}

This section should explain in what ways moisture transport can be quantified and used, give a few examples for each and maybe motivate  why we do it like we want to. 

To computationally study the change of moisture transport it first needs to be quantified. 
The vast majority of literature use some form of vertically integrated humidity, the variants will be explained in the following section.  \todo{Create a table showing how different quantification were used in different algorithms}
The main usage of these algorithms was to find a filamentary structure called \enquote{Atmospheric Rivers}, a prominent way of water vapor transportation in the extratropic regions \cite{gimeno_atmospheric_2014}. 

% See Section \ref{sec:atmo-rivers} for further explanation. 


There are also some notable other algorithms, namely stable oxygen isotope investigation \cite{ma_atmospheric_nodate} and langragian backwards trajectories \cite{zhao_lagrangian_2021}, but both rather look for the origin of the WV instead of its destination and are therefor out of scope for this thesis.

% \subsection{Vertically Integrated Water Vapor (IWV)}

% \subsection{Vertically Integrated Water Vapor Transport (IVT)}

As proposed by \citeauthor{zhu_proposed_1998} in \cite{zhu_proposed_1998}, one way of measuring moisture ($p$) transport is by vertically integrating over the different pressure levels the zonal and meridional fluxes $\overline{pu}$ and $\overline{pv}$. 

An example of using this method can be found in \cite{ayantobo_integrated_2022} with many more references why this method is working well for these kinds of approaches. 

Also, this paper lists some other methods of moisture transportation which are also used

\begin{enumerate}
  \item integrated water vapor distributions
  \item the lagrangian approach
  \item stable oxygen isotope investigation
\end{enumerate}

\subsubsection{Usages of IVT and differences}

In \cite{ralph_dropsonde_2017} they used a vector field of the IVT: $\int_{p_{low}}^{p_{max}} qV dp$, where $p$ is the pressure level, $q$ is the humidity and $V$ the horizontal vector.

In \cite{sousa_north_2020} they used a scalar field based on the euclidian norm of the vector field used by \cite{ralph_dropsonde_2017}.


In \cite{ayantobo_integrated_2022} they also used the euclidian norm on a similar field like \cite{ralph_dropsonde_2017} to measure the impact of  ENSO on south-chinese weather.
% They used it to measure the moisture and displayed the first graphics of the work, illustrating how an IVT map looks at a drought and during a flood.
% Also maps of the mean IVT during each months, displaying the intense raining/wet summers and dry winters

% \subsection{Moisture Budget}

\citeauthor{yang_moisture_2022} showed in their report \cite{yang_moisture_2022} the directions of moisture flux on the continent borders based on the big ERA5 reanalysis.
They measure the moisture based on a equation called the \textit{Moisture Budget}, which is based on multiple Faktors: 



It seems related to the IVT the other authors used, but utilizes the gradient and some other differences. The complete formula is:

$$
\frac{1}{g} \frac{\delta}{\delta t} \int^{P_s}_0 q dp = - \nabla \cdot \frac{1}{g} \int^{P_s}_0 (qv) dp + E - P
$$

With: 

\begin{enumerate}
  \item $p$ is the pressure, $P_s$ is the surface pressure
  \item $q$ is the specific humidity
  \item $v$ is the horizontal wind vector
  \item $E$ is the evaporation
  \item $P$ is the Precipitation
\end{enumerate}


In the actual analysis they used mostly other metrics:


\begin{enumerate}
  \item Vertically integrated Moisture Convergence (\textit{VIMC}): It is basically the gradient of the specific moisture in the air times the Wind vector
  \item $P$ is the precipitation 
  \item $E$ is the evaporation
\end{enumerate}

Furthermore, they evaluated the correlation between the moisture transport and the precipitation variability, which correlate to a significant extent.

%\section{Atmospheric Rivers}
%\label{sec:atmo-rivers}

\todo{I don't know where to put this, maybe it should go into the preliminaries}

This section should explain atmospheric rivers, but since we don't know if they are even relevant so i write it in the end. 

\section{Pattern analysis}

Explain some usages of EOF in data, but extremely important: Explain what \cite{ayantobo_integrated_2022} did since its quite similar. 

% \subsection{Empirical Orthogonal Functions}

See \cite{hannachi_empirical_2007} for a big overview of EOF in atmospheric science.

See \cite{ayantobo_integrated_2022} for a similar approach as we plan it, except it only focuses on the past.
They 
