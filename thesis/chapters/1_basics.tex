
\chapter{Basics}
\label{ch:basics}

This section should explain the basic math to understand the aforementioned topics, not that much needed but still needs to be there.

\section{(Uncertain) Fields}
\label{sec:uncertainfields}

\section{Empirical Orthogonal Functions}
\label{sec:eof}


\subsection{Overview}

Empirical Orthogonal Functions (short: EOFs) analysis, also known as geographically weighted PCA or Proper Orthogonal Decomposition \cite{vietinghoffdiss}, \enquote{is among the most widely and extensively used methods in atmospheric science} \cite{hannachi_empirical_2007}. 
One of its goals is to reduce the usually very high dimensionality of atmospheric data and can be used to link certain modes/patterns to the physics/dynamics of the analyzed system.  
EOFs are a statistical procedure to decompose spatio-temporal data into two components: On the one hand orthogonal spatial patterns, on the other hand corresponding uncorrelated temporal coefficients, representing the activity of their corresponding pattern in certain time steps \cite{hannachi_empirical_2007, vietinghoffdiss}. 
The naming of the components is for from being consistent: The spatial patterns are also called spatial modes or EOFs, while the temporal coefficients are also named principal components (PCs), temporal loadings or EOF coefficients. 
So as a formula, a spatio-temporal field $X(t, s)$ (e.g. a sea level pressure field over time mentioned in Section~\ref{sec:nao}) can be described as

\begin{equation}
  X(t, s) = \sum^{M}_{k=1} c_k(t) u_k(s)
  \label{eq:eof decomposition}
\end{equation}

with $M$ being the number of modes/patterns and  $c_k$ the $k$th temporal coefficients and $u_k$ the spatial pattern \cite{hannachi_empirical_2007}. 

This could be achieved with multiple kinds of patterns, but in practice EOF decomposition tries finding new sets of variables ($c_k(t)$ and $u_k(s)$ from Equation~\ref{eq:eof decomposition}) that each capture a maximum possible amount of variance/variability of the original dataset. 
So the first of $M$ modes captures the most variance, the second one the second most and so on. 

\subsection{Mathematical Derivation and Computation of EOFs}

The goal of this Section is to give an overview of the mathematical origins of EOFs based on the work of \citeauthorwork{hannachi_empirical_2007} as well as their actual practical computation. 
For a more in depth history and derivation, please refer to \cite{hannachi_empirical_2007} and their references, while \citeauthorwork{weiss_tutorial_2019} gives a great hands-on tutorial on POD/EOFs and their interpretation and computation. 

As already explained, the starting point of EOFs is usually a spatio-temporal field $X(t, s)$ defined on a Grid $G$ over $m$ time steps, for example the precipitation analyzed in this Thesis. 
The value at each grid point at geographical location $s_j$ and time $t_i$ is given as $x_{ij}$, with $i = 1, ..., n$  and $j = 1, ..., p$.  
The first step is usually to remove the climatology of the dataset to turn it into anomaly maps. 
The climatology is usually defined as the temporal mean $\bar{x}$ of the analyzed datachunk, so 
\begin{align}
  \bar{x}_i = \frac{1}{n} \sum^{m}_{k=1} x_{ki} \\
  \bar{x} = (\bar{x}_1, ..., \bar{x}_p)^T .
  \label{eq:climatology}
\end{align}

So the values of anomaly maps $x'_{ij}$ at each gridpoint are given as the departure of $X$ from its climatology: 

\begin{equation}
  x'_{ij} = x_{ij} - \bar{x}_j
  \label{eq:anomaly map}
\end{equation}

Since from here on only anomaly maps are considered in the derivation and calcculation of EOFs, the $'$ gets omitted and $X$ stands now for the anomaly data.  

Since geographical data is usually given on a regular 2D grid which depicts the earth's surface, the influence of grid point density (same degree resolution is far more sparse in equatorial regions than in the Arctic) need to be corrected with geographical weights given as a weighting matrix $W$ \cite{vietinghoffdiss}. 

  
