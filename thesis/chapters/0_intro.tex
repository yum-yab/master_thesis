\chapter{Introduction and Motivation}
\label{ch:intro}


\section{Motivation}
\label{sec:motivation}



\section{Climate}
\label{sec:climate}

This section should give an introduction to the current state of climate research. 
Therefor it should explain what the current way of future climate predictions is (Coupled Models), how they work, and 
It should explain some part of the politics, who is involed in what and what the backroud of the most important projects (CMIP, ScenarioMIP \dots). 
It should be explained that the data used is the one that the highest council of fighting climate change uses for its report. 



\section{Research Questions and Thesis Structure}
\label{sec:research_questions}


Structure:

\begin{enumerate}
  \item \textbf{Preliminaries}: explain what climate simulations are, what cmip(6) is and its relation to the IPCC reports and what that means for the global fight against the climate crisis. 
    This chapter should prepare the reader to understand all the related work in Chapter \ref{ch:related_work}.
  \item \textbf{Problem Analysis}: explain what I want to do using the CMIP6 simulations: Describe what the general plan is: Visualization of the moisture transport in Europe with the help. 
    Also define what the goals of the visualizations are: Visualize different scenarios for comparison, visualize uncertainties of different members, visualize evolution over time, also try combining those. 
    Here should be a graphic that explains the workflow that transforms a simulation into some nice pictures
  \item \textbf{Related Work}: Show what efforts have already been done regarding analysis of moisture transport, future and past. 
    Maybe preparing a comparison table would be good. 
  \item \textbf{Realization}: Describe in a step by step way what measures had been taken. 
  \item \textbf{Evaluation}: A little bit unsure how far I (as a CS person) can evaluate this, have to come up with a concept
  \item \textbf{Conclusion}: Same as step before, but there will be something to write about after everything else is written
  
\end{enumerate}

