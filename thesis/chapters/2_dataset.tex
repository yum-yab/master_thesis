\chapter{MPI GE CMIP6}
\label{ch:dataset}

The Max Planck Institute Grand Ensemble CMIP6 (MPI GE CMIP6) is a Single-model initial-condition large ensemble (in short: SMILE) \cite{olonscheck_new_2023}. 
This means that a single model was run with different initial condiditions but the same external forcings (e.g. greenhous gasses) mutiple times ($\Rightarrow$ ensemble). 
This makes it possible to seperate the internal variability from the responses to the external forcing, enabling researchers to better quantify the consequences of climate change (for example) . 
Additionally it makes the research of extreme weather phenomena (e.g. droughts, floods etc.) more robust in spite of their rare occurences \cite{maher_large_2021}. 
As described in Section \ref{sec:climate}, Coupled models 


The dataset chosen for this project is the \textit{Max Planck Institute Grand Ensemble CMIP6} (from now on MPI-GE CMIP6), presented by \citeauthor{olonscheck_new_2023} \cite{olonscheck_new_2023}. 
The reasons for choosing this dataset are manifold:

\begin{enumerate}
  \item It uses the latest (6th) phase of the Coupled Model Intercomparison Project (CMIP6)
  \item Compared to its predecessor (MPI-GE \cite{maher_max_2019}) it provides high frequency output (6 hour intervals vs. monthly means), which enables taking short-lived weather events and structures (e.g. atmospheric rivers) into account which would be lost in the calculation of the mean
  \item 
\end{enumerate}


This section should explain what datasets are available and why I chose the MPI-GE CMIP6 \cite{olonscheck_new_2023}

Maybe but the comparison table from \cite{olonscheck_new_2023} here and expand it a bit. 


