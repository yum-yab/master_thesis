\chapter{MPI GE CMIP6}
\label{ch:dataset}

Explain what I want to do using the CMIP6 simulations: Describe what the general plan is: Visualisation of the moisture transport in Europe with the help . 
Also define what the goals of the visualisations are: Visualize different scenarios for comparison, visualize uncertainties of different members, visualize evolution over time, also try combining those. 
Here should be a graphic that explains the workflow that transforms a simulation into some nice pictures


The dataset chosen for this project is the \textit{Max Planck Institute Grand Ensemble CMIP6} (from now on MPI-GE CMIP6), presented by \citeauthor{olonscheck_new_2023} \cite{olonscheck_new_2023}. 
The reasons for choosing this dataset are manifold:

\begin{enumerate}
  \item It uses the latest (6th) phase of the Coupled Model Intercomparison Project (CMIP6)
  \item Compared to its predecessor (MPI-GE \cite{maher_max_2019}) it provides high frequency output (6 hour intervals vs. monthly means), which enables finding short-lived weather events and structures (e.g. atmospheric rivers)
  \item 
\end{enumerate}


This section should explain what datasets are available and why I chose the MPI-GE CMIP6 \cite{olonscheck_new_2023}

Maybe but the comparison table from \cite{olonscheck_new_2023} here and expand it a bit. 


