\chapter{Methodology}
\label{ch:methodology}

\section{Overview}

Explain what I want to do using the CMIP6 simulations: Describe what the general plan is: Visualisation of the moisture transport in Europe with the help . 
Also define what the goals of the visualisations are: Visualize different scenarios for comparison, visualize uncertainties of different members, visualize evolution over time, also try combining those. 
Here should be a graphic that explains the workflow that transforms a simulation into some nice pictures
\section{Preprocessing}

This should explain how you get a proper IVT field from a CMIP6 simulation, what the caveats are and what calculations are done. 

\begin{itemize}
  \item intitial implementation in julia $\rightarrow$ computation was fast, but the slow IO was a problem 
  \item also very established tools (like CDO) fail 
  \item multiple ideas of optimising the IO:
    \begin{enumerate}
      \item Multiple different libraries - similar results
      \item Parallel reading of different fields - no success or segmentation fault
    \end{enumerate}
    
  \item after asking the helpdesk $\rightarrow$ problem was discovered (few byte transmissions) but not the reason for it 
  \item helpdesk says that large portions of the datasets were never used
  \item after investigating: Turns out Pythons xarray+dask was quite performant
\end{itemize}

