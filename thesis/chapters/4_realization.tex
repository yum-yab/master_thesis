\chapter{Methodology}
\label{ch:methodology}

\section{Overview}

Explain what I want to do using the CMIP6 simulations: Describe what the general plan is: Visualization of the moisture transport in Europe with the help. 
Also define what the goals of the visualizations are: Visualize different scenarios for comparison, visualize uncertainties of different members, visualize evolution over time, also try combining those. 
Here should be a graphic that explains the workflow that transforms a simulation into some nice pictures


\section{Preprocessing}


\subsection{Chosen Framework}
\label{sec:preprocessing_framework}

The goal of this step is to prepare the data for further usage. 
After a few failed attempts with other languages/tools (CDO and Julia, see Section~\ref{sec:preprocessing_problems}), the Python libraries xarray \cite{hoyer_xarray_2017} and dask \cite{rocklin2015dask} were chosen as the fitting tools for this step.
Xarray is a library for handling n-dimensional, labeled arrays. It supports multiple input and output options (amongst others the required NetCDF format) and is compatible with the most popular scientific Python libraries (e.g. Pandas, NumPy). 
It comes with a great variety of features, making it easy to index and transform data and dimensions, joining different datasets (either along a dimension like time or multiple different variables having the same dimensions) and many more. 
But most important, it leverages the Dask library, which enables xarray to actually use the infrastructure of the DKRZ HPC cluster. 
Dask is enabling parallel and out-of-the-core\footnote{This usually means handling datasets larger than RAM, using disks (usually SSDs) as extension for RAM.} computing for the Scientific Python stack. 
Its goal is to be a NumPy clone leveraging the full potential of modern hardware, which usually utilizes multiple computing cores, without the need for rewriting the already existing scientific Python stack. \cite{rocklin2015dask}


\subsection{Process}

One main goal is to reduce the size, so first the geographic region of interest is cut out, which is the Northern Atlantic (derived from \cite{vietinghoff_visual_2021} and \cite{hurrell_overview_2003}): $-90^\circ W - 40^\circ E$, $20^\circ - 80^\circ N$. 
For the compared variables (precipitation, surface pressure) the preprocessing is already done, but in case of IVT it still needs to be calculated. 


\begin{enumerate}
  \item Load four different fields for each time period in each member in each scenario: Specific humidity $hus$, eastward horizontal wind $ua$, northward horizontal wind $va$ and surface pressure $ps$. Here the geographical box around the area of interest is cut out: Europe and the North Atlantic (Longitude: $-90 \to 40$, Latitude: $20 \to 80$, based on \cite{vietinghoff_visual_2021})
  \item For each geographical grid point ($lon$, $lat$) and time step: Calculate the integrals of the product of wind components and specific humidity over the vertical pressure levels $\frac{1}{g}\int_{p_S}^0 hus * ua$ and  $\frac{1}{g}\int_{p_S}^0 hus * va$, with $g$ being the gravitational acceleration ($9.806 \frac{m}{s}$) 
  \item Save the results for each time period in each member in each scenario in a NetCDF file for the further steps. 
\end{enumerate}

The calculations were performed on the high performance computing cluster\footnote{https://docs.dkrz.de/doc/levante/} of the German Climate Calculations Center (DKRZ), due to the MPI GE CMIP6 is saved there and downloading the data would take a lot of time. 
This also result in the goal of this step to minimize the hours on the HPC system since they get billed by the time using nodes. 
Although these steps seem easy, due to the large sizes of the datasets and other issues many challenges were met. 
In the following those will be explained with regard to the step they ocurred in. 



\textbf{1. Data Loading}

\begin{itemize}
  \item cutting out geobox
  \item slow IO $\rightarrow$ started using dask+xarray 
\end{itemize}

\textbf{2. Vertical Integration}

\begin{itemize}
  \item calculate hybrid  sigma pressure levels for each gridpoint and timestep to get the x values for integration
  \item describe my idea of testing the integration
  
\end{itemize}

\subsection{Problems with Preprocessing}
\label{sec:preprocessing_problems}

The steps described in the section before were just the final attempt. 
The first idea was using Climate Data Operators \cite{schulzweida2024}, a command line tool containing multiple operators for processing climate and similar data. 
The operators consist of common statistical and mathematical functions (mean, add, sum), sampling and data selection tools (select geographical or time limits) and other helpful operators like interpolations and even EOF calculation. 
Although this sounded very promising,, it quickly turned out to be very complicated to implement the desired vertical integration in CDO.
The following idea was to implement the IVT calculation in Julia \cite{gao_julia_2020}, using just a NetCDF library \cite{} while the rest was coded from scratch. 
The algorithm was very simple: 

\begin{enumerate}[itemsep=0mm]
  \item Load all datasets into the RAM (as recommended by the NetCDF library itself) and cut out the used geographical limits. This should be feasible since all in all one dataset for one timescope-file accounts for $\sim 12~GB$\footnote{$70~lon * 32~lat * 47~levels * 29220~timesteps * 4~byte \approx 12 GB$}, so the maximum is around $36~GB$, since the surface pressure data is not that large ($\sim 260~MB$)   
  \item Calculate the IVT with trapezoidial integration multithreaded by handling one time\-step by one thread
  \item Write the results (euklidian norm and the meridional/zonal component)
\end{enumerate}

Although julia promises high performance, it performed quite poorly on the HPC.
The reason for this is the slow IO on the cluster: While the calculation itself took only $\sim 235~s$ ($\approx 4~min$)\footnote{Refering here and  in the following to one timescope of 20 years in one member}, the loading of the required datasets took around $\sim 3350~s$ ($\approx 55~min$). 
This results in roughly $5 h$ (including saving the data to disk) for one member of ScenarioMIP, which leads to $250~h$ node hours for one scenario. 
Taking into account that it needs to run for historical simulations as well as other scenarios, this was not feasible according to the limited node hours provided\footnote{Also taking into account that the processes may need to run multiple times due to errors}.

To reduce the loading time of the data multiple optimizations were evaluated. 
First of all, the amount of moved data in memory was minimized by preallocating the needed RAM and writing directly to the preallocated space. 
Furthermore, other NetCDF libraries were tested, but simple loading times were very similar. 
Although this significantly reduced the amount of allocations, the effect on loading time was neglible. 
To actually archive a significant boost in loading time it was tried to load the required datasets (located in different files) in paralllel. 
Unfortunately, the used library \cite{barth_ncdatasetsjl_2024} encountered a segmentation fault used in multiple threads, so the alternative libraries NetCDF.jl and HDF5.jl were explored, since the HDF5 standard allows parrallel access to files \cite{folk_overview_2011}. 
Although the parallel access to files using multiple threads (with HDF5.jl) lead to increased speeds in tests, the results did not yield any significant increased efficiency on the cluster itself.
Even splitting up the loading according to the chunking in the files (all data from one timestep is one chunk) and loading each timestep seperately in one thread even increased the data loading time quite far.  
The next approach was to split the task up into different processes, each one loading data from one variable. This actually reduced time spent to one third in tests, but testing it on the actual data sizes revealed that the $12~GB$ are too much to be returned from the child processes loading the file to the mother process. 

From here on some ather approaches could have been


