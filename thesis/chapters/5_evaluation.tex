\chapter{Results}
\label{ch:results}


\section{Evolution of Patterns}

This Section gives an overview how the EOF patterns change over the time, also comparing the differences of the two chosen climate scenarios, which represent the extemes of climate change handling.   

\subsection{Evolution of Encoded Variability}



\begin{figure}[htb]
  \begin{center}
    \includegraphics[width=0.95\textwidth]{figures/mode_variability_psl_50seasons.png}
  \end{center}
  \caption{Boxplot of the variability encoded in the top five modes of PSL EOF.}\label{fig:psl mode variability}
\end{figure}

The first simple evaluation is to look at the change of share of variability encoded by each EOF (see Equation~\ref{eq:eof variance calculation}). 
The results are displayed in boxplots, with the colored bar being $50\%$  of the members. 
The whiskers are 1.5 the size of the inter-quartile range (distance between upper and lower and of the colored bar), any data point outside that is considered an outlier and represented with dots. 


Figure~\ref{fig:psl mode variability} shows that there is no significant change in the SSP126 scenario in any way. 
The five most significant modes stay pretty much the same across the studied 250-year time period, with the primary mode (NAO) encoding around $39\%$ (median) of the whole dataset variability in each time scope, with fluctuations of the interquartile range ($50\%$ of the data) introduced by the members of the simulations being around $\pm 2\%$, with no significant trend over the years. 
The secondary mode (EAP) median stays around $17\%$, with the quartiles being $\pm 1\%$. 
The median variability ecoded by EOFs 3,4 and 5 is around $13\%$, $8\%$, and $5\%$, respectively. 
Comparing it to the SSP585 scenario, it is obvious that there is very little to no change in Modes 3-5 and 1. 
But interestingly, the median variability encoded by the secondary mode rises from the $17\%$ in the 1850 - 1900 scope to around $20\%$ in the last one, exposing a clear trend over the course of climate change.   


\begin{figure}[hbt]
  \begin{center}
    \includegraphics[width=0.95\textwidth]{figures/mode_variability_ivt_50seasons.png}
  \end{center}
  \caption{Same as Figure~\ref{fig:psl mode variability} but with IVT}\label{fig:ivt mode variability}
\end{figure}

The same analysis with the IVT patterns (Figure~\ref{fig:ivt mode variability}) reveal a general upwards trend in the primary mode of IVT, from median $26\%$ in the first window to around $28\%$ in the last. 
This trend is very similar in both SSP126 and SSP 585. 
Modes 3,4 and 5 also look very similar in both evaluated scenarios, with a median encoded variability of $7\%$, $5\%$, and $3\%$. 
Similar to Figure~\ref{fig:psl mode variability}, the secondary mode (representing around $15\%$ of variability) shows upward trend in scenario SSP585 to around $17\%$, which is not recognizable in the SSP126 scenario. 

\begin{figure}[htb]
  \begin{center}
    \includegraphics[width=0.95\textwidth]{figures/mode_variability_pr_50seasons.png}
  \end{center}
  \caption{Same as Figure~\ref{fig:psl mode variability} but with precipitation}\label{fig:pr mode variability}
\end{figure}




\section{Relationships with other Variables}
