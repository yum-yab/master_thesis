\begin{center}
  \textsc{Abstract}
\end{center}
%
\noindent
%
The distribution and variability of precipitation in Europe are significantly influenced by moisture transport over the north(east)ern Atlantic.
Due to the turbulent nature of moisture transport, structural changes are difficult to track, which is tackled by analyzing the main variability patterns with a sliding window approach.
The objective of this thesis is to visually analyze and compare changes in different future climate scenarios. 
In addition, we investigated connections with dominant Atlantic oscillation patterns (\acf{nao} and \acf{eap}) and precipitation in Europe.
Based on the latest \acl{mpige6}, visualizing the variability introduced by the 50 members of the simulation poses a challenge.
To mitigate the visual clutter associated with the representation of multiple members' contour lines, a hexbin-based approach was used to facilitate the analysis of variability introduced by these numerous members.
The results identified two dominant modes of water vapor transport, which demonstrated considerable stability across different members and time periods, along with structural changes in several spatial patterns. 
In general, the variability explained by all moisture transport variability patterns increases, especially in pronounced climate change scenarios. 
This effect was also observed in the primary pattern of precipitation and \ac{eap}.
The modes of moisture transport also exhibited significant correlations with the leading oscillation and precipitation patterns. 

