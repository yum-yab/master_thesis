\documentclass{mimosis}

\usepackage{metalogo}

%%%%%%%%%%%%%%%%%%%%%%%%%%%%%%%%%%%%%%%%%%%%%%%%%%%%%%%%%%%%%%%%%%%%%%%%
% Some of my favourite personal adjustments
%%%%%%%%%%%%%%%%%%%%%%%%%%%%%%%%%%%%%%%%%%%%%%%%%%%%%%%%%%%%%%%%%%%%%%%%
%
% These are the adjustments that I consider necessary for typesetting
% a nice thesis. However, they are *not* included in the template, as
% I do not want to force you to use them.

% This ensures that I am able to typeset bold font in table while still aligning the numbers
% correctly.
\usepackage{etoolbox}

% Add todonotes for writing process
\usepackage{todonotes}

% use libertinus font 
\usepackage{libertinus}

% for tables 
\usepackage{tabularray}
% for useful symbols
\usepackage{gensymb}

% for compact lists
\usepackage{enumitem}



%%%%%%%%%%%%%%%%%%%%%%%%%%%%%%%%%%%%%%%%%%%%%%%%%%%%%%%%%%%%%%%%%%%%%%%%
% Hyperlinks & bookmarks
%%%%%%%%%%%%%%%%%%%%%%%%%%%%%%%%%%%%%%%%%%%%%%%%%%%%%%%%%%%%%%%%%%%%%%%%

\usepackage[%
  colorlinks = true,
  citecolor  = RoyalBlue,
  linkcolor  = RoyalBlue,
  urlcolor   = RoyalBlue,
  unicode,
  ]{hyperref}

\usepackage{bookmark}

% for better footnotes
\usepackage{cleveref}
\crefformat{footnote}{#2\footnotemark[#1]#3}
%%%%%%%%%%%%%%%%%%%%%%%%%%%%%%%%%%%%%%%%%%%%%%%%%%%%%%%%%%%%%%%%%%%%%%%%
% Bibliography
%%%%%%%%%%%%%%%%%%%%%%%%%%%%%%%%%%%%%%%%%%%%%%%%%%%%%%%%%%%%%%%%%%%%%%%%
%
% I like the bibliography to be extremely plain, showing only a numeric
% identifier and citing everything in simple brackets. The first names,
% if present, will be initialized. DOIs and URLs will be preserved.

\usepackage[%
  autocite     = plain,
  backend      = biber,
  doi          = true,
  url          = false,
  giveninits   = true,
  hyperref     = true,
  maxbibnames  = 99,
  maxcitenames = 2,
  sortcites    = true,
  style        = numeric,
  ]{biblatex}

\input{bibliography-mimosis}
\addbibresource{zotero_lib.bib}
\addbibresource{custom_references.bib}

%%%%%%%%%%%%%%%%%%%%%%%%%%%%%%%%%%%%%%%%%%%%%%%%%%%%%%%%%%%%%%%%%%%%%%%%
% Fonts
%%%%%%%%%%%%%%%%%%%%%%%%%%%%%%%%%%%%%%%%%%%%%%%%%%%%%%%%%%%%%%%%%%%%%%%%

% change font for headings 

% set sans font to libertinus sans
% \renewcommand{\sfdefault}{Libertinus Sans}
% \titleformat*{\section}{\Large\sffamily}
% \titleformat*{\subsection}{\large\sffamily}
% \titleformat*{\subsubsection}{\normalsize\sffamily}


% set main font to libertinus Serif
% \setmainfont{Libertinus Serif}

% \setsansfont{IBM Plex Sans}
% \setmonofont{IBM Plex Mono}
% \ifxetexorluatex
%   \usepackage{unicode-math}
%   \setmainfont{Cambria}
%   \setmathfont{Cambria Math}
% 
%   % Load some missing symbols from another font.
%   \setmathfont{STIX Two Math}[%
%     range = {
%       \sharp,
%       \natural,
%       \flat,
%       \clubsuit,
%       \spadesuit,
%       \checkmark
%     }
%   ]
%   \setmonofont[Scale=MatchLowercase]{Source Code Pro}
% \else
%   \usepackage[lf]{ebgaramond}
%   \usepackage[oldstyle,scale=0.7]{sourcecodepro}
%   \singlespacing
% \fi

\newacronym[description={Principal component analysis}]{PCA}{PCA}{principal component analysis}
\newacronym[description={Empirical Orthogonal Functions}]{EOF}{EOF}{empirical orthogonal functions}
\newacronym                                            {SNF}{SNF}{Smith normal form}
\newacronym[description={Topological data analysis}]   {TDA}{TDA}{topological data analysis}
\newacronym                                            {GCM}{GCM}{Global Coupled Model}

\newglossaryentry{LaTeX}{%
  name        = {\LaTeX},
  description = {A document preparation system},
  sort        = {LaTeX},
}

\newglossaryentry{Real numbers}{%
  name        = {$\real$},
  description = {The set of real numbers},
  sort        = {Real numbers},
}

\makeindex
\makeglossaries

%%%%%%%%%%%%%%%%%%%%%%%%%%%%%%%%%%%%%%%%%%%%%%%%%%%%%%%%%%%%%%%%%%%%%%%%
% Custom Commands
%%%%%%%%%%%%%%%%%%%%%%%%%%%%%%%%%%%%%%%%%%%%%%%%%%%%%%%%%%%%%%%%%%%%%%%%

\newcommand{\citeauthorwork}[1]{\citeauthor{#1} \cite{#1}}

%%%%%%%%%%%%%%%%%%%%%%%%%%%%%%%%%%%%%%%%%%%%%%%%%%%%%%%%%%%%%%%%%%%%%%%%
% Ordinals
%%%%%%%%%%%%%%%%%%%%%%%%%%%%%%%%%%%%%%%%%%%%%%%%%%%%%%%%%%%%%%%%%%%%%%%%

\makeatletter
\@ifundefined{st}{%
  \newcommand{\st}{\textsuperscript{\textup{st}}\xspace}
}{}
\@ifundefined{rd}{%
  \newcommand{\rd}{\textsuperscript{\textup{rd}}\xspace}
}{}
\@ifundefined{nd}{%
  \newcommand{\nd}{\textsuperscript{\textup{nd}}\xspace}
}{}
\makeatother

\renewcommand{\th}{\textsuperscript{\textup{th}}\xspace}

% make only sections and chapters appear in TOC 
\setcounter{tocdepth}{1}
%%%%%%%%%%%%%%%%%%%%%%%%%%%%%%%%%%%%%%%%%%%%%%%%%%%%%%%%%%%%%%%%%%%%%%%%
%  Actual Document
%%%%%%%%%%%%%%%%%%%%%%%%%%%%%%%%%%%%%%%%%%%%%%%%%%%%%%%%%%%%%%%%%%%%%%%%


\begin{document}


\frontmatter
  

\newcommand{\titleOfThesis}{Analyzing the Evolution of Moisture Transport Patterns in the North Atlantic based on Ensemble Simulations}
\newcommand{\kindOfThesis}{Masterarbeit} % Bachelor or Master Thesis

\newcommand{\myuniversity}{Universität Leipzig}
\newcommand{\myuni}{Universität Leipzig} % short affiliation of the OvGU
\newcommand{\myunistreet}{Augustusplatz 10}
\newcommand{\myunizipcity}{04109 Leipzig, Germany}
\newcommand{\myschool}{Fakultät für Mathematik und Informatik}  % German
\newcommand{\mydepartment}{Institut für Informatik}
\newcommand{\myfirstname}{Denis}
\newcommand{\mylastname}{Streitmatter}
\newcommand{\mybirthdate}{30.\ December 1997}
\newcommand{\mybirthplace}{Kösching}

% infai/kilt info
\newcommand{\mycompany}{Institut für Angewandte Informatik}
\newcommand{\myworkinggroup}{Abteilung für Bild und Signalverarbeitung}
\newcommand{\mycompanystreet}{Goerdelerring 9}
\newcommand{\mycompanyzipcity}{04109 Leipzig, Germany}


\newcommand{\mydegree}{B.\ Sc.\ Computer Science}
\newcommand{\myplace}{Leipzig}
\newcommand{\myyear}{2021}

\begin{titlepage}
  \begin{center}
  {\LARGE \textbf{\myuniversity}}\\[5mm]
% \begin{figure*}[h]
% \hbox{}\hfill
% \begin{minipage}[t]{9cm}
%     		 \centering
%     		 \includegraphics[width=4cm]{logos/uni-leipzig-logo.png}
% \end{minipage}
% \hfill\hbox{}
% \end{figure*}
  {\Large \myschool}\\
  {\large \mydepartment}\\[20mm]
		%{\large\bf \titleOfThesis}\\[10mm]
		{\Large\textbf \titleOfThesis}\\[10mm]
    {\LARGE \kindOfThesis}\\[10mm]
  %{\small Author:}\\[1mm]
  %{\large \myfirstname~\mylastname}\\[5mm]
  %{\small \today}\\[15mm]
  
  
% Needs to fit: http://studium.fmi.uni-leipzig.de/fileadmin/Studienbuero/documents/Formulare/Deckblatt_Dipl_BSc_MSc_Arbeit_.rtf
    
    
\vspace{5cm}
% \begin{tabular}{p{.6\textwidth}p{.5\textwidth}}
%     \multicolumn{1}{l}{submitted at \today}  & \multicolumn{1}{r}{submitted by} \\[3mm]
%     & \multicolumn{1}{r}{{\large \myfirstname~\mylastname}} \\
%     & \multicolumn{1}{r}{Informatik Bsc.}
% \end{tabular}




%   \vspace{1cm}
%   \renewcommand{\arraystretch}{.9}
%   \begin{tabular}{cc}
% 	\multicolumn{2}{c}{\small Advisers:} \\[1mm]
% 	{\small Supervisor} & {\small Supervisor} \\
% 	{\large Johannes Frey, M.Sc.} & {\large Dr.-Ing. Sebastian Hellmann} \\[2mm]
% 	{\small \myworkinggroup} 			 & {\small \myworkinggroup} \\
% 	{\small \mycompany} 			 & {\small \mycompany} 	\\
% 	{\small \mycompanystreet} 		 & {\small \mycompanystreet} \\
% 	{\small \myunizipcity}		 & {\small \myunizipcity} 
%   \end{tabular}
  
		\renewcommand{\arraystretch}{1}
  \end{center}
  
Leipzig, August 2024\hfill vorgelegt von \\[3mm]
\hspace*{\fill} {\large \myfirstname~\mylastname}\\
\hspace*{\fill} Studiengang Master Informatik

\vspace{1cm}

{\large\textbf{Betreuende Hochschullehrer:}}
%
% \quad {\large Dr. Baldwin Nsonga}
%
% \quad \myuni, \myworkinggroup
%

\quad {\large Prof. Dr. Gerik Scheuermann}

\quad \myuni, \myworkinggroup
%  \vspace*{5cm}
%  \makeatletter
%  \begin{center}
%    \begin{Huge}
%      \@title
%    \end{Huge}\\[0.1cm]
%    %
%    \begin{Large}
%      \@subtitle
%    \end{Large}\\
%    %
%    \emph{by}\\
%    \@author  \mydegree
%    %
%    \vfill
%    A document submitted in partial fulfillment
%    of the requirements for the degree of\\
%    \emph{Technical Report}\\
%    at\\
%    \textsc{Miskatonic University}
%  \end{center}
  \makeatother
\end{titlepage}

\newpage
\null
\thispagestyle{empty}
\newpage

  %\begin{titlepage}
  \thispagestyle{titlepage}
  \begin{center}
  {\LARGE \textbf{\myuniversity}}\\[5mm]
%   \begin{figure*}[h]
%   \hbox{}\hfill
%   \begin{minipage}[t]{9cm}
% 			 \centering
% 			 \includegraphics[width=4cm]{logos/uni-leipzig-logo.png}
%   \end{minipage}
%   \hfill\hbox{}
%   \end{figure*}
  {\Large \myschool}\\
  {\large \mydepartment}\\[20mm]
		%{\large\bf \titleOfThesis}\\[10mm]
		{\Large\bf \titleOfThesis}\\[10mm]
    {\LARGE \kindOfThesis}\\[10mm]
  %{\small Author:}\\[1mm]
  %{\large \myfirstname~\mylastname}\\[5mm]
  %{\small \today}\\[15mm]
  
  
% Needs to fit: http://studium.fmi.uni-leipzig.de/fileadmin/Studienbuero/documents/Formulare/Deckblatt_Dipl_BSc_MSc_Arbeit_.rtf
    
    
\vspace{5cm}
% \begin{tabular}{p{.6\textwidth}p{.5\textwidth}}
%     \multicolumn{1}{l}{submitted at \today}  & \multicolumn{1}{r}{submitted by} \\[3mm]
%     & \multicolumn{1}{r}{{\large \myfirstname~\mylastname}} \\
%     & \multicolumn{1}{r}{Informatik Bsc.}
% \end{tabular}




%   \vspace{1cm}
%   \renewcommand{\arraystretch}{.9}
%   \begin{tabular}{cc}
% 	\multicolumn{2}{c}{\small Advisers:} \\[1mm]
% 	{\small Supervisor} & {\small Supervisor} \\
% 	{\large Johannes Frey, M.Sc.} & {\large Dr.-Ing. Sebastian Hellmann} \\[2mm]
% 	{\small \myworkinggroup} 			 & {\small \myworkinggroup} \\
% 	{\small \mycompany} 			 & {\small \mycompany} 	\\
% 	{\small \mycompanystreet} 		 & {\small \mycompanystreet} \\
% 	{\small \myunizipcity}		 & {\small \myunizipcity} 
%   \end{tabular}
  
		\renewcommand{\arraystretch}{1}
  \end{center}
  
Leipzig, März 2021\hfill vorgelegt von \\[3mm]
\hspace*{\fill} {\large \myfirstname~\mylastname}\\
\hspace*{\fill} Studiengang Bachelor Informatik

\vspace{1cm}

{\large \textbf{Betreuende Hochschullehrer:}}

{\large Dr.-Ing. Sebastian Hellmann }\\
\mycompany/KILT

{\large Johannes Frey, M.Sc. }\\
\mycompany/KILT

{\large Dr. Eric Peukert}\\
Abteilung Datenbanken Universität Leipzig
  
\end{titlepage}

\thispagestyle{empty}
\vspace*{\fill}
\begin{minipage}{.95\textwidth}
\textbf{\mylastname,~\myfirstname:}\\
\emph{\titleOfThesis}\\
\kindOfThesis, \myuni \\
\myplace, \myyear.
\end{minipage}

\cleardoublepage

% =============================================================================

%%%%%%%%%%%%%%%%%%%%%%%%%%%%%%%%%%%%%%%%%%%%%%%%%%%%%%%%%%%%%%%%%%%%%%%%%%%%%
%%% Inhaltsverzeichnis
%%%%%%%%%%%%%%%%%%%%%%%%%%%%%%%%%%%%%%%%%%%%%%%%%%%%%%%%%%%%%%%%%%%%%%%%%%%%%

\setcounter{tocdepth}{2}
\tableofcontents

%\addcontentsline{toc}{chapter}{Abbildungsverzeichnis}
%\listoffigures

%\addcontentsline{toc}{chapter}{Tabellenverzeichnis}
%\listoftables

%\addcontentsline{toc}{chapter}{Algorithmenverzeichnis}
%\listofalgorithms
%\todo{Anleitung Algorithmen schreiben: \url{http://en.wikibooks.org/wiki/LaTeX/Algorithms}}

\cleardoublepage
% =============================================================================

\phantomsection
\section*{Abstract}
\addcontentsline{toc}{chapter}{Abstract}
Over the last years, a huge amount of work has been done to improve the ability of machines to utilize data on the Web. One approach is the Semantic Web, using ontologies as a way to make the knowledge of a domain machine-usable. Even though many ontologies were developed and published, a unified system to handle those has not surfaced, leaving consumers as well as publishers to deal with many uncertainties and challenges. 

This thesis presents DBpedia Archivo, an augmented ontology archive. It discovers, crawls, versions, and archives ontologies available on the Web. Each version of them is persisted on the DBpedia Databus. Additionally, Archivo augments the ontologies with different tests and features. 
The goals of Archivo are to provide a backup service for ontology-versions as well as to encourage publishers to follow best practices. 
For this Archivo rates the ontologies with a star system, making problems visible at a glance. A comparison to existing, similar systems is given.

\cleardoublepage
% =============================================================================


%\thispagestyle{plain}
%\phantom{.}
%\vspace{70mm}

%\begin{center}
%	\todo[inline]{This section is optional! It is basically an motivational cite for this work as it can be found in many books. Example is provided}
%	\textit{
%		\vspace{0.5cm}
%		The validation of clustering structures is \\
%		the most difficult and frustrating part of cluster analysis. \\ 
%		\vspace{0.5cm}
%		Without a strong effort in this direction, \\
%		cluster analysis will remain a black art 
%		accessible only to those \\
%		true believers who have experience and great courage.}
%\end{center}
%\begin{flushright}
%	\citet{Jain1988}
%	Anil K. Jain and Richard C. Dubes
%\end{flushright}

%\uselengthunit{mm}
%textwidth=\printlength{\textwidth}\\
%textheight=\printlength{\textheight}\\
%top=\printlength{\top}\\

%\cleardoublepage
% =============================================================================


%\thispagestyle{plain}
%\section*{Acknowledgements}
%\todo[inline]{This section is optional!}

%\cleardoublepage
% =============================================================================


  \begin{center}
  \textsc{Abstract}
\end{center}
%
\noindent
%
The distribution and variability of precipitation in Europe are significantly influenced by moisture transport over the north(east)ern Atlantic. The objective of my thesis is to analyze the evolution of moisture transport patterns in various future climate scenarios. The foundation of this research lies in the MPI-GE, the Max Planck Institute Grand Ensemble Dataset, comprising an ensemble of 100 members for different RCP (climate) scenarios up until 2100. Each member provides multiple fields of relevant climate data.

Atmospheric rivers (ARs) play a substantial role in moisture transport and are identified using integrated water vapor transport (IVT). The analysis employs Empirical Orthogonal Functions (EOFs) to examine changes in spatial patterns of ARs across different climate scenarios and time periods.


  \tableofcontents

\mainmatter
  \chapter{Introduction and Motivation}
\label{ch:intro}


\section{Motivation}
\label{sec:motivation}


Since the discovery (and further confirmation) of the greenhouse effect in the years from 1824 to 1900 \cite{fourier1824remarques, foote1856circumstances} humans came a long way of fighting the consequences of the increased greenhouse gas concentration in earth's atmosphere. 
In 1972 \citeauthor{sawyer1972man} summarized the knwoledge and predicted quite accurately the warming at the end of the century \cite{sawyer1972man}
Especially the last decades the climate crisis gained more and more attention, leading to the creation of multiple international organizations and institutions (e.g. the International Panel on Climate Change (IPCC) in 1988).


\begin{figure}[t]
  \begin{center}
    \includegraphics[width=0.95\textwidth]{figures/ipcc_6th_report_impacts_climate_change.png}
  \end{center}
  \caption{Impact of Climate Change for Humans, taken from \cite{lee2024climate}}
  \label{fig:impacts_climate_change}
\end{figure}



In 2019 more than 11,000  scientists from around the world released a declaration \cite{ripple_world_2019}, calling governments from around the world to action.
The mid and long-term consequences are manyfold and go far beyond the general rising of the worlds' average temperature (see FIgure \ref{fig:impacts_climate_change}), e.g. shifts in circulation systems like the North Atlantic Oscillation (NAO) \cite{vietinghoff_visual_2021}, which in turn also have varying consequences. 
Understanding what consequences may lay ahead of us is a crucial step in tackling these challanges, and this thesis aims to follow up on the research of \citeauthor{vietinghoff_visual_2021}, trying to evaluate in a similar manner the systemic changes of moisture transport patterns in Europe and the northern Atlantic. 


\section{Climate and Climate Research}
\label{sec:climate}

This section should give an introduction to the current state of climate research. 
Therefor it should explain what the current way of future climate predictions is (Coupled Models), how they work, and 
It should explain some part of the politics, who is involed in what and what the backroud of the most important projects (CMIP, ScenarioMIP \dots). 
It should be explained that the data used is the one that the highest council of fighting climate change uses for its report. 

\begin{enumerate}
  \item
  
\end{enumerate}

\textbf{IPCC and the Coupled Model Intercomparison Project (CMIP)}


The reason for the endorsement of the IPCC by the UN General Assembly 1988 was to prepare comprehensive reviews and report about the current state of scientific knowledge and research. 
Since then there were six assement cycles and six reports were published, condensing the research of the scientific community. Figure \ref{fig:impacts_climate_change} is a graphic from the latest report for policy makers from 2023 \cite{lee2024climate}, displaying the probable consequences for humans in climate change.
A main source for such figures in the reports are so-called Global Coupled Models (GCM), trying to model the state and evolution of certain fields of earth data \todo{back up with sources and better writing}. 
They consist of multiple Models, each representing a major part of Earth's complex climate system (like atmosphere, hydrosphere, etc.), also allowing to model the dynamic interactions between these parts. 
In the mid 90s the Coupled Model Intercomparison Project (CMIP) was brought to life, with the aim of streamlining results of GCMs and making them compareable. 
CMIP provides the outer structure, amongst others what kind of simulations to produce (e.g. preindustrial control simulations, future scenarios etc.), what kind of resolutions to provide and also how these results should be serialized.
Since then the results of CMIP played a major part in the reports of the IPCC \cite{touzepeiffer_coupled_2020}, and are now even called \enquote{... one of the foundational elements of climate science} \cite{eyring_overview_2016}. 
CMIP is currently in its 6th phase, corresponding to the recently finsihed 6th Assessment Report of the IPCC \cite{lee2024climate}. 



\textbf{The North Atlantic Oscillation}




 
\section{Research Questions and Thesis Structure}
\label{sec:research_questions}


Structure:

\begin{enumerate}
  \item \textbf{Preliminaries}: explain what climate simulations are, what cmip(6) is and its relation to the IPCC reports and what that means for the global fight against the climate crisis. 
    This chapter should prepare the reader to understand all the related work in Chapter \ref{ch:related_work}.
  \item \textbf{Problem Analysis}: explain what I want to do using the CMIP6 simulations: Describe what the general plan is: Visualization of the moisture transport in Europe with the help. 
    Also define what the goals of the visualizations are: Visualize different scenarios for comparison, visualize uncertainties of different members, visualize evolution over time, also try combining those. 
    Here should be a graphic that explains the workflow that transforms a simulation into some nice pictures
  \item \textbf{Related Work}: Show what efforts have already been done regarding analysis of moisture transport, future and past. 
    Maybe preparing a comparison table would be good. 
  \item \textbf{Realization}: Describe in a step by step way what measures had been taken. 
  \item \textbf{Evaluation}: A little bit unsure how far I (as a CS person) can evaluate this, have to come up with a concept
  \item \textbf{Conclusion}: Same as step before, but there will be something to write about after everything else is written
  
\end{enumerate}


  
\chapter{Basics}
\label{ch:basics}

This section should explain the basic math to understand the aforementioned topics, not that much needed but still needs to be there.

\section{(Uncertain) Fields}
\label{sec:uncertainfields}

\subsection{Grids, Scalar Fields and Interpolation}

\todo{Also add description of contour lines}

\subsection{Map Projections}

\subsection{Uncertain Fields}

\section{Empirical Orthogonal Functions}
\label{sec:eof}


\subsection{Overview}

Empirical Orthogonal Functions (short: EOFs) analysis, also known as geographically weighted PCA or Proper Orthogonal Decomposition \cite{vietinghoffdiss}, \enquote{is among the most widely and extensively used methods in atmospheric science} \cite{hannachi_empirical_2007}. 
One of its goals is to reduce the usually very high dimensionality of atmospheric data and can be used to link certain modes/patterns to the physics/dynamics of the analyzed system.  
EOFs are a statistical procedure to decompose spatio-temporal data into two components: On the one hand orthogonal spatial patterns, on the other hand corresponding uncorrelated temporal coefficients, representing the activity of their corresponding pattern in certain time steps \cite{hannachi_empirical_2007, vietinghoffdiss}. 
The naming of the components is for from being consistent: The spatial patterns are also called spatial modes, PC loadings, EOFs or even sometimes PCs, while the temporal coefficients are also named principal components (PCs), EOF amplitudes or EOF (expansion) coefficients \cite{hannachi_empirical_2007}. 
So as a formula, a spatio-temporal field $X(t, s)$ (e.g. a sea level pressure field over time mentioned in Section~\ref{sec:nao}) can be described as

\begin{equation}
  X(t, s) = \sum^{M}_{k=1} c_k(t) u_k(s)
  \label{eq:eof decomposition}
\end{equation}

with $M$ being the number of modes/patterns and  $c_k$ the $k$th temporal coefficients and $u_k$ the spatial pattern \cite{hannachi_empirical_2007}. 

This could be achieved with multiple kinds of patterns, but in practice EOF decomposition tries finding new sets of variables ($c_k(t)$ and $u_k(s)$ from Equation~\ref{eq:eof decomposition}) that each capture a maximum possible amount of variance/variability of the original dataset. 
So the first of $M$ modes captures the most variance, the second one the second most and so on. 

\subsection{Mathematical Derivation and Computation of EOFs}

The goal of this Section is to give an overview of the mathematical origins of EOFs based on the work of \citeauthorwork{hannachi_empirical_2007} as well as their actual practical computation. 
For a more in depth history and derivation, please refer to \cite{hannachi_empirical_2007} and their references, while \citeauthorwork{weiss_tutorial_2019} gives a great hands-on tutorial on POD/EOFs and their interpretation and computation. 

As already explained, the starting point of EOFs is usually a spatio-temporal field $X(t, s)$ defined on a Grid $G$ over $n$ time steps, for example the precipitation analyzed in this Thesis. 
The value at each grid point at geographical location $s_j$ and time $t_i$ is given as $x_{ij}$, with $i = 1, ..., n$  and $j = 1, ..., p$.  
The first step is usually to remove the climatology of the dataset to turn it into anomaly maps. 
The climatology is usually defined as the temporal mean $\bar{x}$ of the analyzed datachunk, so 
\begin{align}
  \bar{x}_i = \frac{1}{n} \sum^{n}_{k=1} x_{ki} \\
  \bar{x} = (\bar{x}_1, ..., \bar{x}_p)^T .
  \label{eq:climatology}
\end{align}

So the values of anomaly maps $x'_{ij}$ at each grid point are given as the departure of $X$ from its climatology: 



\begin{equation}
  x'_{ij} = x_{ij} - \bar{x}_j \\
  \label{eq:anomaly map elements}
\end{equation}

And so the final anomaly map $X'$ is: 
\begin{equation}
  X' = \begin{pmatrix}
x'_{11} & x'_{12} & \cdots & x'_{1j} \\
x'_{21} & x'_{22} & \cdots & x'_{2j} \\
\vdots & \vdots & \ddots & \vdots \\
x'_{i1} & x'_{i2} & \cdots & x'_{ij}
\end{pmatrix}
  \label{eq:anomaly map}
\end{equation}

% Since from here on only anomaly maps are considered in the derivation and calculation of EOFs, the $'$ gets omitted and $X$ stands now for the anomaly data.  
The first usual step for generating EOFs is the covariance matrix defined by 

\begin{equation}
  S = \frac{1}{n} X'^T X' 
  \label{eq:covariance map}
\end{equation}

This covariance matrix with the values $s_{ab}$ with $a,b = 1, \cdots, p$ contains the covariance of any grid point with any other grid point over the time. 
To find EOFs means determining a unit length direction $u = (u_1, \cdots, u_p)$ that explains the most variability. 
This problem is therefor equivalent to the solution to the eigenvalue problem, so finding all the eigenvectors ($\equiv$ EOFs) and their eigenvalue. Which means that the vector $u$ multiplied by the covariance matrix $S$ is equivalent to the multiplication with a scalar $\lambda^2$ (the eigenvalue):  

\begin{equation}
  Su = \lambda^2 u
  \label{eq:eigenvalue problem} 
\end{equation}

So to find the $k$th EOF of a Covariance matrix, the eigenvectors $u$ are sorted by the (largest first) value of their corresponding eigenvalue $\lambda^2$. 
The primary (or dominant) EOF the first in this order, the secondary EOF the second and so on. 
The variance $v_k$ of the original dataset associated with the $k$th EOF can then be calculated with: 

\begin{equation}
  v_k = \frac{\lambda^2_k}{\sum^{p}_{i=1} \lambda^2_i}
  \label{eq:eof variance calculation}
\end{equation}

The temporal coefficients can then in turn be calculated projecting the eigenvectors $u_k$ on the original anomaly map $X'$ with: 

\begin{equation}
  a_{k} = X'u_k
\end{equation}

Together they fulfill the requirements of the decomposition in Equation~\ref{eq:eof decomposition}. 
% So the coefficients in Equation~\ref{eq:eof decomposition}
Note here that the solutions being eigenvectors means that the multiplication by any scalar $\alpha$ (i.e. $\alpha u_k$ and $\alpha^{-1} a_k$) is also a valid solution to the problem. 
This leaves room of choosing scale and direction in a useful way (see Section~\ref{sec:eof_calc} for a practical implementation) \cite{vietinghoffdiss}. 

\subsection{Calculation and Application to the geographical Domain}


Since geographical data is usually given on a regular 2D grid which depicts the earth's surface, the influence of grid point density (same degree resolution is far more sparse in equatorial regions than in the Arctic) need to be corrected with geographical weights. 
Those can be approximated by the square root of the cosine of the respective latitude \cite{hannachi_primer_nodate, vietinghoffdiss} with a similar diagonal matrix as depicted in \cite{hannachi_primer_nodate}: 

\begin{equation}
  W = \begin{pmatrix}
    \cos(\theta_1) & 0 & \cdots & 0 \\
0 & \cos(\theta_2) & \cdots & 0 \\
\vdots & \vdots & \ddots & \vdots \\
0 & 0 & \cdots & \cos(\theta_p)
\end{pmatrix}
  \label{eq:geographical weighting}
\end{equation}
  

Fortunately, there is no need to calculate the covariance matrix and solve the eigenvalue problem. 
Linear Algebra provides a tool called \textit{Singular Value Decomposition} (SVD), which decomposes any matrix $X$ into three components: 

\begin{equation}
  X = L \Lambda R^T 
  \label{eq:svd definition}
\end{equation}

$L$ contains the left singular vectors, $R$ the right singular vectors and $\Lambda$ a diagonal matrix containing the singular values $\lambda_k$ (as used in Equation~\ref{eq:eof variance calculation} above). 

Now all of the above is used to calculate the EOFs of geographical data by applying SVD to the matrix (like in \citeauthorwork{vietinghoffdiss}): 

\begin{equation}
  \tilde{X} = \frac{1}{\sqrt{n - 1}} W^{\frac{1}{2}} X' 
  \label{eq:complete data preperation}
\end{equation}


When using SVD of $\tilde{X}$ with time as the first dimension (like depicted here), the columns of $R^T$ are the EOFs (so $u_k(s)$ of Equation~\ref{eq:eof decomposition}) and the columns of $L$ multiplied with $\sqrt{n - 1}$ are the principal components or EOF coefficients ($c_k(t)$ in Equation~\ref{eq:eof decomposition}).
As explained above, this result can be scaled, which is explained in detail in Section~\ref{sec:eof_calc}.
\todo{go over this whole section and check everything for being correct!!!!}




  \chapter{MPI GE CMIP6}
\label{ch:dataset}

The Max Planck Institute Grand Ensemble CMIP6 (MPI GE CMIP6) is a Single-model initial-condition large ensemble (in short: SMILE) \cite{olonscheck_new_2023}. 
This means that a single model was run with different initial condiditions but the same external forcings (e.g. greenhous gasses) mutiple times ($\Rightarrow$ ensemble). 
This makes it possible to seperate the internal variability from the responses to the external forcing, enabling researchers to better quantify the consequences of climate change (for example) . 
Additionally it makes the research of extreme weather phenomena (e.g. droughts, floods etc.) more robust in spite of their rare occurences \cite{maher_large_2021}. 
As described in Section \ref{sec:climate}, Coupled models 


The dataset chosen for this project is the \textit{Max Planck Institute Grand Ensemble CMIP6} (from now on MPI-GE CMIP6), presented by \citeauthor{olonscheck_new_2023} \cite{olonscheck_new_2023}. 
The reasons for choosing this dataset are manifold:

\begin{enumerate}
  \item It uses the latest (6th) phase of the Coupled Model Intercomparison Project (CMIP6)
  \item Compared to its predecessor (MPI-GE \cite{maher_max_2019}) it provides high frequency output (6 hour intervals vs. monthly means), which enables taking short-lived weather events and structures (e.g. atmospheric rivers) into account which would be lost in the calculation of the mean
  \item 
\end{enumerate}


This section should explain what datasets are available and why I chose the MPI-GE CMIP6 \cite{olonscheck_new_2023}

Maybe but the comparison table from \cite{olonscheck_new_2023} here and expand it a bit. 

\section{Future Scenarios}

While not in its core, CMIP6 also defines multiple different scenarions with different forcings. 


  \chapter{Related Work}

\section{Climate simulation datasets}

General infos from \cite{mpige}:

\begin{itemize}
  \item 
	
\end{itemize}

\subsection{RCP Scenarios}

\subsection{Questions arising about using climate simulation datasets}

\begin{itemize}
  \item How many ensemble members are needed for a correct assessment?
  \item How to sort them out? Random?
  \item 
\end{itemize}

\subsection{MPI-GE - The Max Planck Institute grand Ensemble}

In \cite{mpige} theere is much inforamtion available:

\subsection{CMIP5 - Coupled Model Intercomparison Project}

In \cite{taylor2012overview_cmip5}


\section{Precipitation Literature}

\subsection{Saisonality in Precipitation variability}


The work of \citeauthor{precipitation_seasonality}

\section{Means of moisture transport}

\subsection{vertically integrated water vapor transport}

As proposed by \citeauthor{AProposedAlgorithmforMoistureFluxesfromAtmosphericRivers} in \cite{AProposedAlgorithmforMoistureFluxesfromAtmosphericRivers}, one way of measuring moisture ($p$) transport is by vertically integrating over the different pressure levels the zonal and meridional fluxes $\overline{pu}$ and $\overline{pv}$. 

An example of using this method can be found in \cite{Ayantobo2021IntegratedMT} with many more references why this method is working well for these kinds of approaches. 
Also this paper lists some other methods of moisture transportation which are also used:

\begin{enumerate}
  \item integrated water vapor distributions (see \cite{gimeno2014atmospheric_rivers_review})
  \item the lagrangian approach
  \item stable oxygen isotope investigation
\end{enumerate}



\section{Pattern analysis}

\subsection{Empirical Orthogonal Functions}

See \cite{hannachi2007eof_review} for a big overview of EOF in atmospheric science.

See \cite{Ayantobo2021IntegratedMT} for an similar approach as we plan it, except it only focuses on the past

  \chapter{Methodology}
\label{ch:methodology}

\section{Overview}

Explain what I want to do using the CMIP6 simulations: Describe what the general plan is: Visualisation of the moisture transport in Europe with the help . 
Also define what the goals of the visualisations are: Visualize different scenarios for comparison, visualize uncertainties of different members, visualize evolution over time, also try combining those. 
Here should be a graphic that explains the workflow that transforms a simulation into some nice pictures


\section{Preprocessing}

The goal of this step is to prepare the data for further usage. 
One main goal is to reduce the size, so first of all the geographic region of interest is cut out, which is the Northern Atlantic (derived from \cite{vietinghoff_visual_2021}): $-90^\circ W - 40^\circ E$, $20^\circ - 80^\circ N$. 
For the compared variables (preciptation, surface pressure) the preprocessing is already done, but in case of IVT it still needs to be calculated. 


\begin{enumerate}
  \item Load four different fields for each time period in each member in each szenario: Specific humidity $hus$, eastward horizontal wind $ua$, northward horizontal wind $va$ and surface pressure $ps$. Here the geographical box around the area of interest is cut out: Europe and the North Atlantic (Longitude: $-90 \to 40$, Latitude: $20 \to 80$, based on \cite{vietinghoff_visual_2021})
  \item For each geographical gridpoint ($lon$, $lat$) and timestep: Calculate the integrals of hte product of wind components and specific humidity over the vertical pressure levels $\frac{1}{g}\int_{p_S}^0 hus * ua$ and  $\frac{1}{g}\int_{p_S}^0 hus * va$, with $g$ being the gravitational acceleration ($9.806 \frac{m}{s}$) 
  \item Save the results for each time period in each member in each szenario in a NetCDF file for the further steps. 
\end{enumerate}

The calculations were performed on the high performance computing cluster\footnote{https://docs.dkrz.de/doc/levante/} of the German Climate Calculations Center (DKRZ), due to the MPI GE CMIP6 is saved there and downloading the data would take a lot of time. 
This also result in the goal of this step to minimize the hours on the HPC system since they get billed by the time using nodes. 
Although these steps seem easy, due to the large sizes of the datasets and other issues many challenges were met. 
In the following those will be explained with regard to the step they ocurred in. 



\textbf{1. Data Loading}

\begin{itemize}
  \item cutting out geobox
  \item slow IO $\rightarrow$ started using dask+xarray 
\end{itemize}

\textbf{2. Vertical Integration}

\begin{itemize}
  \item calculate hybrid  sigma pressure levels for each gridpoint and timestep to get the x values for integration
  \item describe my idea of testing the integration
  
\end{itemize}

\subsection{Problems with Preprocessing}

The steps described in the section before were just the final attempt. 
The first idea was using Climate Data Operators \cite{schulzweida2024}, a command line tool containing multiple operators for processing climate and similar data. 
The operators consist of common statistical and mathematical functions (mean, add, sum), sampling and data selection tools (select geographical or time limits) and other helpful operators like interpolations and even EOF calculation. 
Although this sounded very promising,, it quickly turned out to be very complicated to implement the desired vertical integration in CDO.
The following idea was to implement the IVT calculation in Julia \cite{gao_julia_2020}, using just a NetCDF library \cite{} while the rest was coded from scratch. 
The algorithm was very simple: 

\begin{enumerate}[itemsep=0mm]
  \item Load all datasets into the RAM (as recommended by the NetCDF library itself) and cut out the used geographical limits. This should be feasible since all in all one dataset for one timescope-file accounts for $\sim 12~GB$\footnote{$70~lon * 32~lat * 47~levels * 29220~timesteps * 4~byte \approx 12 GB$}, so the maximum is around $36~GB$, since the surface pressure data is not that large ($\sim 260~MB$)   
  \item Calculate the IVT with trapezoidial integration multithreaded by handling one time\-step by one thread
  \item Write the results (euklidian norm and the meridional/zonal component)
\end{enumerate}

Although julia promises high performance, it performed quite poorly on the HPC.
The reason for this is the slow IO on the cluster: While the calculation itself took only $\sim 235~s$ ($\approx 4~min$)\footnote{Refering here and  in the following to one timescope of 20 years in one member}, the loading of the required datasets took around $\sim 3350~s$ ($\approx 55~min$). 
This results in roughly $5 h$ (including saving the data to disk) for one member of ScenarioMIP, which leads to $250~h$ node hours for one scenario. 
Taking into account that it needs to run for historical simulations as well as other scenarios, this was not feasible according to the limited node hours provided\footnote{Also taking into account that the processes may need to run multiple times due to errors}.

To reduce the loading time of the data multiple optimizations were evaluated. 
First of all, the amount of moved data in memory was minimized by preallocating the needed RAM and writing directly to the preallocated space. 
Furthermore, other NetCDF libraries were tested, but simple loading times were very similar. 
Although this significantly reduced the amount of allocations, the effect on loading time was neglible. 
To actually archive a significant boost in loading time it was tried to load the required datasets (located in different files) in paralllel. 
Unfortunately, the used library \cite{barth_ncdatasetsjl_2024} encountered a segmentation fault used in multiple threads, so the alternative libraries NetCDF.jl and HDF5.jl were explored, since the HDF5 standard allows parrallel access to files \cite{folk_overview_2011}. 
Although the parallel access to files using multiple threads (with HDF5.jl) lead to increased speeds in tests, the results did not yield any significant increased efficiency on the cluster itself.
Even splitting up the loading according to the chunking in the files (all data from one timestep is one chunk) and loading each timestep seperately in one thread even increased the data loading time quite far.  
The next approach was to split the task up into different processes, each one loading data from one variable. This actually reduced time spent to one third in tests, but testing it on the actual data sizes revealed that the $12~GB$ are too much to be returned from the child processes loading the file to the mother process. 

From here on some ather approaches could have been



  \chapter{Results}
\label{ch:results}


\section{Evolution of Patterns}
\label{sec:pattern evolution}

This Section gives an overview how the EOF patterns change over the time, also comparing the differences of the two chosen climate scenarios, which represent the extremes of climate change handling.   

\subsection{Evolution of Encoded Variability}



\begin{figure}[htb]
  \begin{center}
    \includegraphics[width=0.85\textwidth]{figures/mode_variability_psl_50seasons.png}
  \end{center}
  \caption{Boxplot of the variability encoded in the top five modes of PSL EOF.}\label{fig:psl mode variability}
\end{figure}

The first simple evaluation is to look at the change of share of variability encoded by each EOF (see Equation~\ref{eq:eof variance calculation}). 
The results are displayed in boxplots, with the colored bar being $50\%$  of the members. 
The whiskers are 1.5 the size of the interquartile range (distance between upper and lower and of the colored bar), any data point outside that is considered an outlier and represented with dots. 


Figure~\ref{fig:psl mode variability} shows that there is no significant change in the SSP126 scenario in any way. 
The five most significant modes stay pretty much the same across the studied 250-year time period, with the primary mode (NAO) encoding around $39\%$ (median) of the whole dataset variability in each time scope, with fluctuations of the interquartile range ($50\%$ of the data) introduced by the members of the simulations being around $\pm 2\%$, with no significant trend over the years. 
The secondary mode (EAP) median stays around $17\%$, with the quartiles being $\pm 1\%$. 
The median variability encoded by EOFs 3,4 and 5 is around $13\%$, $8\%$, and $5\%$, respectively. 
Comparing it to the SSP585 scenario, it is obvious that there is very little to no change in Modes 3-5 and 1. 
But interestingly, the median variability encoded by the secondary mode rises from the $17\%$ in the 1850 - 1900 scope to around $20\%$ in the last one, exposing a clear trend over the course of climate change.   


\begin{figure}[hbt]
  \begin{center}
    \includegraphics[width=0.85\textwidth]{figures/mode_variability_ivt_50seasons.png}
  \end{center}
  \caption{Same as Figure~\ref{fig:psl mode variability} but with IVT}\label{fig:ivt mode variability}
\end{figure}

The same analysis with the IVT patterns (Figure~\ref{fig:ivt mode variability}) reveal a general upwards trend in the primary mode of IVT, from median $26\%$ in the first window to around $28\%$ in the last. 
This trend is very similar in both SSP126 and SSP 585. 
Modes 3,4 and 5 also look very similar in both evaluated scenarios, with a median encoded variability of $8\%$, $6\%$, and $5\%$. 
Similar to Figure~\ref{fig:psl mode variability}, the secondary mode (representing around $15\%$ of variability) shows upward trend in scenario SSP585 to around $17\%$, which is not recognizable in the SSP126 scenario. 

\begin{figure}[htb]
  \begin{center}
    \includegraphics[width=0.85\textwidth]{figures/mode_variability_pr_50seasons.png}
  \end{center}
  \caption{Same as Figure~\ref{fig:psl mode variability} but with precipitation}\label{fig:pr mode variability}
\end{figure}

The comparison of mode variability evolution of precipitation EOFs (Figure~\ref{fig:pr mode variability}) shows no significant changes of modes 3,4, and 5 between both evaluated scenarios. 
Those encode on median $5\%$, $6\%$ and $6.5\%$ with small fluctuations introduced by the members. 
Mode 2 also looks very similar in both scenarios, with a median encoded variability of around $8.5\%$. 
The primary EOF on the other shows significant differences across scenarios: While it has a far greater variability across members then the other modes and follows a general upwards trend in both SSP126 and SSP585, it is more pronounced in the latter. 
It evolves from around $12.5\%$ in the 1850-1900 window to around $14\%$ in SSP126 and $15.5\%$ in SSP585.  


\subsection{Evolution of Spatial Patterns}

This Section shows the evolution of the spatial EOF patterns, shown in Figure~\ref{fig:5modes each variable}. 
Since the EOF modes four and five are generally quite low and similar in their eigenvalues (which directly correspond to the variance (see Equation~\ref{eq:eof variance calculation}) encoded shown in Figures~\ref{fig:ivt mode variability}, \ref{fig:psl mode variability}, and \ref{fig:pr mode variability}), they are left out of the analysis of this and the following sections, as modes' eigenvalues need to be well separated from each other \cite{hannachi_empirical_2007}. 
Usually, the rule of thumb introduced by \citeauthorwork{north_sampling_1982} is used, but since the eigenvalues of the first three modes (or two for precipitation, see Figure~\ref{fig:pr mode variability}), this is left out of this Thesis. 

The variability introduced through the 50 members of the MPI GE CMIP6 is displayed here with the hexbin approach explained in Section~\ref{sec:vis_analysis}, while a discussion of the hexbin visualization compared to the classical spaghetti is given in Section~\ref{sec:discussion}.
The images in this section are the first and last frame from the video displaying the evolution of the different scenarios.\todo{Reference the videos somehow}  


\begin{figure}[htb]
  \begin{center}
    \includegraphics[width=0.85\textwidth]{figures/ivt_spat_patterns_hexbin_18501899_ssp585_50seasons.png}
    \includegraphics[width=0.85\textwidth]{figures/ivt_spat_patterns_hexbin_20502100_ssp126_50seasons.png}
    \includegraphics[width=0.85\textwidth]{figures/ivt_spat_patterns_hexbin_20502100_ssp585_50seasons.png}
  \end{center}
  \caption{The top three EOFs of IVT data, with a 50 winter scope and hexbins visualizing the variability introduced by simulation members. The top row displays the state in the historical simulation (second half of 19th century), while middle (SSP126) and bottom (SSP585) display the state in the second half of the 21st century. The red line shows the contour line of zero of the preindustrial control simulation. }\label{fig:ivt eof evolution}
\end{figure}

Figure~\ref{fig:ivt eof evolution} shows the evolution of IVT EOF spatial patterns. 
In general, regardless of the future scenario, EOF1 and EOF2 stays structurally very stable across all the ensembles' members, which can be seen on the clear, dark green borders of the colored surfaces. 
EOF3 on the other hand seems pretty unstable, since most of the map is covered in light greed hexagons, which means that the contour lines of zero switch significantly between all the 50 members of the ensemble. 
This also has consequences for the alignment across members and time, which will obviously not work if the patterns differ greatly across time and members. 
Therefor, the analysis regarding such patterns will be kept short since the multi-member, sliding window analysis of such patterns used in this Thesis does not apply very well to such patterns. 

The dominant EOF1 pattern of IVT is characterized by a positive IVT values reaching from Florida to Spain and negative values from the USA east coast to Northern Europe and the Northern Atlantic. 
There are three, clearly visible borders of these positive and negative areas, associated with three groups of contour lines: The first going through Canada, quite coherently across members (many dark green hexagons in a row), and then following the east coast of Greenland, fading out over the mainland quite differently across the members (many light green hexagons in a larger area). 
The second border follows a similar pattern: Starting quite coherent across members at the beginning of the Florida peninsula, over the Atlantic to the east coast of France, and then fading out differently in Eastern Europe. 
The third border goes through the most southern part of the evaluated area, through North Africa and then to the Arabian Peninsula, staying pretty consistent across its path. 
Comparing the state of the patterns at the end of the different future scenario simulations, the change is quite subtle but visible in the area of France and the South of the British Islands:
While the dark green hexagons in the beginning of the historical simulation are on/below the red line of the preindustrial simulation, the majority of zero contour lines in SSP126 seem to be above the preindustrial control contour line. 
In SSP585, the dark green hexagons stretch even further north, indicating that the slight northward shift of IVT EOF1 at the end of the 21st century is even more pronounced. 

The IVT EOF2 
% While the border in Greenland looks very similar at the end of SSP126 and the begin of the historical simulation, 
% On this example a way of interpreting the hexbin visualization is given, which can be applied to the patterns in Figures \ref{fig:pr eof evolution} and \ref{fig:psl eof evolution}. 




\begin{figure}
  \begin{center}
    \includegraphics[width=0.95\textwidth]{figures/pr_spat_patterns_hexbin_18501899_ssp585_50seasons.png}
    \includegraphics[width=0.95\textwidth]{figures/pr_spat_patterns_hexbin_20502100_ssp126_50seasons.png}
    \includegraphics[width=0.95\textwidth]{figures/pr_spat_patterns_hexbin_20502100_ssp585_50seasons.png}
  \end{center}
  \caption{Same as Figure~\ref{fig:ivt eof evolution}, but with precipitation data.}\label{fig:pr eof evolution}
\end{figure}

\begin{figure}
  \begin{center}
    \includegraphics[width=0.95\textwidth]{figures/psl_spat_patterns_hexbin_18501899_ssp585_50seasons.png}
    \includegraphics[width=0.95\textwidth]{figures/psl_spat_patterns_hexbin_20502100_ssp126_50seasons.png}
    \includegraphics[width=0.95\textwidth]{figures/psl_spat_patterns_hexbin_20502100_ssp585_50seasons.png}
  \end{center}
  \caption{Same as Figure~\ref{fig:ivt eof evolution}, but with sea level pressure data.}\label{fig:psl eof evolution}
\end{figure}

\section{Relationships with other Variables}



\subsection{Relationships of EOFs}


\subsubsection{Relationships of IVT and Sea Surface Pressure EOFS (NAO and EAP)}

\begin{figure}
  \begin{center}
    \includegraphics[width=0.95\textwidth]{figures/crosscorrelation_boxplot_ivt_psl_modes11_ssp126_50seasons.png}
    \includegraphics[width=0.95\textwidth]{figures/crosscorrelation_boxplot_ivt_psl_modes11_ssp585_50seasons.png}
  \end{center}
  \caption{Maximum cross correlation of Mode 1 of Sea Level Pressure and Mode 1 of IVT. On the right are images of the corresponding spatial patterns. Lag is a boxplot and indicates the shift in time used for finding the maximum (absolute) cross correlation.}
  \label{fig:crosscor ivt psl modes11}
\end{figure}

\begin{figure}
  \begin{center}
    \includegraphics[width=0.95\textwidth]{figures/crosscorrelation_boxplot_ivt_psl_modes12_ssp126_50seasons.png}
    \includegraphics[width=0.95\textwidth]{figures/crosscorrelation_boxplot_ivt_psl_modes12_ssp585_50seasons.png}
  \end{center}
  \caption{Same as Figure~\ref{fig:crosscor ivt psl modes11}, but with mode 2 of Sea Level Pressure (EAP).}
  \label{fig:crosscor ivt psl modes12}
\end{figure}

\begin{figure}
  \begin{center}
    \includegraphics[width=0.95\textwidth]{figures/crosscorrelation_boxplot_ivt_psl_modes22_ssp126_50seasons.png}
    \includegraphics[width=0.95\textwidth]{figures/crosscorrelation_boxplot_ivt_psl_modes22_ssp585_50seasons.png}
  \end{center}
  \caption{Same as Figure~\ref{fig:crosscor ivt psl modes11}, but with mode 2 of Sea Level Pressure (EAP) and IVT.}
  \label{fig:crosscor ivt psl modes22}
\end{figure}


\begin{figure}
  \begin{center}
    \includegraphics[width=0.95\textwidth]{figures/crosscorrelation_boxplot_ivt_psl_modes21_ssp126_50seasons.png}
    \includegraphics[width=0.95\textwidth]{figures/crosscorrelation_boxplot_ivt_psl_modes21_ssp585_50seasons.png}
  \end{center}
  \caption{Same as Figure~\ref{fig:crosscor ivt psl modes11}, but with mode 2 of IVT EOF.}
  \label{fig:crosscor ivt psl modes21}
\end{figure}






\subsubsection{Relationships of Precipitation and IVT EOFs}

\begin{figure}
  \begin{center}
    \includegraphics[width=0.95\textwidth]{figures/crosscorrelation_boxplot_pr_ivt_modes11_ssp126_50seasons.png}
    \includegraphics[width=0.95\textwidth]{figures/crosscorrelation_boxplot_pr_ivt_modes11_ssp585_50seasons.png}
  \end{center}
  \caption{Same as Figure~\ref{fig:crosscor ivt psl modes11}, but with modes 1 of precipitation and IVT EOFs.}\label{fig:crosscor pr ivt modes11}
\end{figure}

\begin{figure}
  \begin{center}
    \includegraphics[width=0.95\textwidth]{figures/crosscorrelation_boxplot_pr_ivt_modes22_ssp126_50seasons.png}
    \includegraphics[width=0.95\textwidth]{figures/crosscorrelation_boxplot_pr_ivt_modes22_ssp585_50seasons.png}
  \end{center}
  \caption{Same as Figure~\ref{fig:crosscor pr ivt modes11}, but with modes 2 of each variables' EOFs.}\label{fig:crosscor pr ivt modes22}
\end{figure}


\subsubsection{Relationships of Precipitation and Sea Level Pressure EOFs}



\begin{figure}
  \begin{center}
    \includegraphics[width=0.95\textwidth]{figures/crosscorrelation_boxplot_psl_pr_modes11_ssp126_50seasons.png}
    \includegraphics[width=0.95\textwidth]{figures/crosscorrelation_boxplot_psl_pr_modes11_ssp585_50seasons.png}
  \end{center}
  \caption{Same as Figure~\ref{fig:crosscor ivt psl modes11}, but with Sea Level Pressure and precipitation EOFs.}\label{fig:crosscor psl pr modes11}
\end{figure}

\begin{figure}
  \begin{center}
    \includegraphics[width=0.95\textwidth]{figures/crosscorrelation_boxplot_psl_pr_modes22_ssp126_50seasons.png}
    \includegraphics[width=0.95\textwidth]{figures/crosscorrelation_boxplot_psl_pr_modes22_ssp585_50seasons.png}
  \end{center}
  \caption{Same as Figure~\ref{fig:crosscor ivt psl modes11}, but with modes 2 of both Sea Level Pressure and precipitation EOFs.}\label{fig:crosscor psl pr modes22}

\end{figure}




\subsection{Relationships of EOFs with Variables}


\section{Discussion of Interpretation}
\label{sec:discussion}

  \chapter{Conclusions and Future Work}
\label{ch:conclusions}

\section{Conclusions}
\label{sec:conclusions}



\section{Future Work}
\label{sec:FutureWork}

\begin{itemize}
  \item More differenst stats to compare: spearman correlation, regression ....
  \item Measure significance of modes (e.g. how much of spains precipitation is explained through one mode)

\end{itemize}


% This ensures that the subsequent sections are being included as root
% items in the bookmark structure of your PDF reader.
\bookmarksetup{startatroot}
\backmatter

  \begingroup
    \let\clearpage\relax
    \glsaddall
    \printglossary[type=\acronymtype]
    \newpage
    \printglossary
  \endgroup

  \printindex
  \printbibliography

\end{document}
